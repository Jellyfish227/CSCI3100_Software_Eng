\documentclass[a4paper,11pt]{scrartcl}
\usepackage{releasenotes} % Your custom style
\usepackage{tabularx}
\usepackage{booktabs}
\usepackage{enumitem}
\usepackage{hyperref}

\title{Release Notes\\Kaiju Academy}
\author{Group A2}
\date{\today}

\begin{document}

\maketitle

\section*{Release Notes}
\addcontentsline{toc}{section}{Release Notes}

\begin{tabularx}{\textwidth}{l l l X}
\toprule
Version & Release Date & Author(s) & Summary \\
\midrule
1.1 & 2025-05-07 & Group A2 & Major release: stable public version, payment/credit system integrated, improved code assessment, and educator dashboards. \\
1.0 & 2025-04-12 & Group A2 & Initial complete implementation: user registration, courses, code runner, basic educator/student features. \\
0.5 & 2025-03-12 & Group A2 & Beta milestone: core backend and frontend features, code execution, user roles. \\
0.2 & 2025-03-05 & Group A2 & Alpha: database schema, initial authentication, minimal course browsing. \\
0.1 & 2024-02-20 & Group A2 & Project setup: repository, pipeline, initial UI prototyping. \\
\bottomrule
\end{tabularx}

\vspace{1.5em}

\section{What's New in Version 1.1}

\begin{itemize}[leftmargin=*]
    \item \textbf{Credit and Payment System:} Users can purchase credits and enroll in paid courses. All payment handled securely via third-party gateway.
    \item \textbf{Improved Code Assessment:} Enhanced code editor with auto-grading, feedback, and support for multiple languages.
    \item \textbf{Educator Dashboard:} Educators can manage courses, track student progress, and upload materials efficiently.
    \item \textbf{UI/UX Enhancements:} More responsive design, improved accessibility, and modernized navigation.
    \item \textbf{Security Improvements:} MFA support, better error handling, and encrypted user data.
    \item \textbf{Performance Optimizations:} Faster page load, scalable backend, and robust deployment on AWS.
\end{itemize}

\section{Fixed Issues}

\begin{itemize}[leftmargin=*]
    \item Fixed bugs in JWT authentication and session expiry.
    \item Resolved UI glitches in course browsing and profile pages.
    \item Improved error messages for failed logins and invalid course actions.
    \item Addressed concurrency issues in code execution and grading.
\end{itemize}

\section{Known Issues}

\begin{itemize}[leftmargin=*]
    \item Limited browser support for some interactive editor features; recommend Chrome or Firefox for best experience.
    \item Mobile UI in beta; minor layout issues may occur on certain devices.
    \item Community/forum features are not included in this release.
    \item Licence management requirement is documented in SRS and Design, but not implemented in this release.
\end{itemize}



\section{Upgrade Instructions}

\begin{enumerate}[leftmargin=*]
    \item Backup your database and user data before upgrading.
    \item Deploy the new backend Lambda functions and synchronize the SurrealDB schema.
    \item Update the frontend application and environment configurations.
    \item For production, ensure all environment variables (e.g., \texttt{JWT\_SECRET}, AWS keys) are set.
    \item Test core flows: registration, login, course enrollment, code submission, and educator dashboard access.
\end{enumerate}

\section{Credits}

\begin{itemize}[leftmargin=*]
    \item \textbf{YU Ching Hei:} Backend, authentication, API design
    \item \textbf{Lei Hei Tung:} Frontend, UI/UX, educator dashboard
    \item \textbf{Ankhbayar Enkhtaivan:} Error handling, code assessment, documentation
    \item \textbf{Yum Ho Kan:} Frontend, user profile, testing
    \item \textbf{Leung Chung Wang:} Documentation, deployment, configuration
\end{itemize}

\section{Acknowledgments}
This document was prepared with the assistance of AI tools (e.g., ChatGPT 4.1) for drafting and review.


\end{document}