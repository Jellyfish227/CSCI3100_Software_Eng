\documentclass[a4paper, 11pt]{scrreprt}
\usepackage{srs}  % Load SRS document style

% Document-specific settings
\hypersetup{
    pdftitle={Software Requirement Specification},
    pdfauthor={C. H. Yu},
    pdfsubject={Technical Documentation},
    pdfkeywords={LaTeX, SRS, Software Requirements Specification, Software}
}

\def\myversion{0.2}
\def\projectname{Kaiju Academy}
\date{}

\begin{document}

\pagenumbering{roman}  % Start with roman numerals for front matter

\begin{titlepage}
    \begin{flushright}
        \rule{\textwidth}{5pt}\vskip1cm
        \begin{bfseries}
            \Huge{SOFTWARE REQUIREMENTS\\ SPECIFICATION}\\
            \vspace{1.6cm}
            for\\
            \vspace{1.6cm}
            \projectname\\  % Use project name variable
            \vspace{1.6cm}
            \LARGE{Version \myversion}\\
            \vspace{1.6cm}
            Prepared by\\
            Group A2\\
            \vspace{0.5cm}
            \begin{tabularx}{\textwidth}{X l r}
            YU Ching Hei & 1155193237 & \href{mailto:chyu@link.cuhk.edu.hk}{chyu@link.cuhk.edu.hk}\\
            Lei Hei Tung & 1155194969 & \href{mailto:1155194969@link.cuhk.edu.hk}{1155194969@link.cuhk.edu.hk}\\
            Yum Ho Kan & 1155195234 & \href{mailto:1155195234@link.cuhk.edu.hk}{1155195234@link.cuhk.edu.hk}\\
            <name> & <student id> & <email>\\
            <name> & <student id> & <email>\\
            <name> & <student id> & <email>\\
            \end{tabularx}\\
            \vspace{1.6cm}
            The Chinese University of Hong Kong\\
            Department of Computer Science and Engineering\\
            CSCI3100: Software Engineering\\
            \vspace{1.6cm}
            \today\\
        \end{bfseries}
    \end{flushright}
\end{titlepage}

\setuptoc{toc}{totoc}  % Add ToC to itself using KOMA-Script
\tableofcontents

\addchap{Document Revision History}  % Unnumbered chapter that appears in ToC

\begin{center}
    \begin{tabularx}{\textwidth}{>{\raggedright\arraybackslash}p{2cm}>{\raggedright\arraybackslash}p{3cm}>{\raggedright\arraybackslash}p{3cm}>{\raggedright\arraybackslash}X}
        \toprule
        Version & Revised By & Revision Date & Comments\\
        \midrule
        0.1 & C. H. Yu & 2025-02-08 & \begin{revisionitem}[Updated:]
            \item Initial document structure
            \item Basic template setup
        \end{revisionitem}\\
        \midrule
        0.2 & C. H. Yu & 2025-02-08 & \begin{revisionitem}[Updated:]
            \item Formatting
        \end{revisionitem}\\
        \midrule
        0.3 & C. H. Yu & 2025-02-08 & \begin{revisionitem}[Updated:]
            \item Titlepage formatting
            \item Page numbering
            \item Chapters title formatting
            \item Chapter and sections arrangement
        \end{revisionitem}\\
        \midrule
        0.4 & H. T. Lei & 2025-02-09 & \begin{revisionitem}[Added:]
            \item Specific requirements
        \end{revisionitem}\\
        \midrule
        0.5 & H. T. Lei & 2025-02-09 & \begin{revisionitem}[Added:]
            \item Other requirements
        \end{revisionitem}\\
        \midrule
        0.5 & H. K. Yum & 2025-02-10 & \begin{revisionitem}[Updated:]
            \item Nonfunctional Requirements
        \end{revisionitem}\\
        \bottomrule
    \end{tabularx}
\end{center}

\clearpage
\pagenumbering{arabic}  % Switch to arabic numbers for main content

\chapter{Introduction}

\section{Document Purpose}
$<$Identify the product whose software requirements are specified in this document, 
including the revision or release number. Describe the scope of the product that is 
covered by this SRS, particularly if this SRS describes only part of the system or a 
single subsystem.$>$
TODO: Write 1-2 paragraphs describing the purpose of this document as explained above

\section{Project Scope}
$<$Provide a short description of the software being specified and its purpose, including relevant benefits, objectives, and goals. 
TODO: 1-2 paragraphs describing the scope of the product. Make sure to describe the benefits associated with the product.
$>$

\section{Intended Audience and Document Overview}
$<$Describe the different types of reader that the document is intended for, such as developers, project managers, marketing staff, users, testers, and documentation writers (In your case it would probably be the "client" and the professor). Describe what the rest of this SRS contains and how it is organized. Suggest a sequence for reading the document, beginning with the overview sections and proceeding through the sections that are most pertinent to each reader type.$>$

\section{Definitions, Acronyms and Abbreviations}
$<$Define all the terms necessary to properly interpret the SRS, including acronyms and abbreviations. You may wish to build a separate glossary that spans multiple projects or the entire organization, and just include terms specific to a single project in each SRS.
TODO: Please provide a list of all abbreviations and acronyms used in this document sorted in alphabetical order
$>$

\section{Document Conventions}
$<$In general this document follows the IEEE formatting requirements. Use Arial font size 11, or 12 throughout the document for text. Use italics for comments. Document text should be single spaced and maintain the 1" margins found in this template. For Section and Subsection titles please follow the template. 

TODO: Describe any standards or typographical conventions that were followed when writing this SRS, such as fonts or highlighting that have special significance. Sometimes, it is useful to divide this section to several sections, e.g., Formatting Conventions, Naming Conventions, etc
$>$

\section{References and Acknowledgments}
$<$List any other documents or Web addresses to which this SRS refers. These may 
include user interface style guides, contracts, standards, system requirements 
specifications, use case documents, or a vision and scope document. Provide 
enough information so that the reader could access a copy of each reference, 
including title, author, version number, date, and source or location.$>$

\chapter{Overall Description}

\section{Product Perspective}
$<$Describe the context and origin of the product being specified in this SRS.  
For example, state whether this product is a follow-on member of a product 
family, a replacement for certain existing systems, or a new, self-contained 
product. If the SRS defines a component of a larger system, relate the 
requirements of the larger system to the functionality of this software and 
identify interfaces between the two. A simple diagram that shows the major 
components of the overall system, subsystem interconnections, and external 
interfaces can be helpful.$>$

\section{Product Functionality}
$<$Summarize the major functions the product must perform or must let the user 
perform. Details will be provided in Section 3, so only a high level summary 
(such as a bullet list) is needed here. Organize the functions to make them 
understandable to any reader of the SRS. A picture of the major groups of 
related requirements and how they relate, such as a top level data flow diagram 
or object class diagram, is often effective.$>$

\section{User Classes and Characteristics}
$<$Identify the various user classes that you anticipate will use this product.  
User classes may be differentiated based on frequency of use, subset of product 
functions used, technical expertise, security or privilege levels, educational 
level, or experience. Describe the pertinent characteristics of each user class.  
Certain requirements may pertain only to certain user classes. Distinguish the 
most important user classes for this product from those who are less important 
to satisfy.$>$

\section{Operating Environment}
$<$Describe the environment in which the software will operate, including the 
hardware platform, operating system and versions, and any other software 
components or applications with which it must peacefully coexist.$>$ 

\section{Design and Implementation Constraints}
$<$Describe any items or issues that will limit the options available to the 
developers. These might include: corporate or regulatory policies; hardware 
limitations (timing requirements, memory requirements); interfaces to other 
applications; specific technologies, tools, and databases to be used; parallel 
operations; language requirements; communications protocols; security 
considerations; design conventions or programming standards (for example, if the 
customer's organization will be responsible for maintaining the delivered 
software).$>$

\section{Assumptions and Dependencies}
$<$List any assumed factors (as opposed to known facts) that could affect the 
requirements stated in the SRS. These could include third-party or commercial 
components that you plan to use, issues around the development or operating 
environment, or constraints. The project could be affected if these assumptions 
are incorrect, are not shared, or change. Also identify any dependencies the 
project has on external factors, such as software components that you intend to 
reuse from another project, unless they are already documented elsewhere (for 
example, in the vision and scope document or the project plan).$>$

\section{User Documentation}
$<$List the user documentation components (such as user manuals, on-line help, 
and tutorials) that will be delivered along with the software. Identify any 
known user documentation delivery formats or standards.$>$


\chapter{Specific Requirements}

\section{External Interface Requirements}
\subsection{User Interfaces}
$<$Describe the logical characteristics of each interface between the software 
product and the users. This may include sample screen images, any GUI standards 
or product family style guides that are to be followed, screen layout 
constraints, standard buttons and functions (e.g., help) that will appear on 
every screen, keyboard shortcuts, error message display standards, and so on.  
Define the software components for which a user interface is needed. Details of 
the user interface design should be documented in a separate user interface 
specification.$>$

Kaiju Academy will have a graphical user interface (GUI) designed for ease of use and accessibility across different roles, including students, educators, and administrators. The UI components include:
\begin{description}
    \item[$\cdot$ Dashboard:] Upon login, users will be presented with a dashboard displaying their enrolled courses, upcoming deadlines, progress tracking, and notifications.
    \item[$\cdot$ Navigation:] A global navigation bar providing quick access to Courses, Assessments, Calendar, Discussion Forum, and Profile settings.
    \item[$\cdot$ Course Pages:] Each course contains a structured layout displaying modules, videos, PDFs, quizzes, and assessments with a progress tracker.
    \item[$\cdot$ Assessment Interface:] Interactive assessment screens allowing multiple-choice (MC) questions with self-checking, short-answer autograded questions, and long-answer questions submitted for manual grading.
    \item[$\cdot$ Progress Tracking:] A visual roadmap for students to track completed and pending modules and overall course progress.
    \item[$\cdot$ Discussion Forum:] A interactive discussion forum with search, filter, and sort options.
    \item[$\cdot$ Calendar:] An integrated calendar highlighting course schedules, assignment deadlines, and upcoming events.
    \item[$\cdot$ Notifications:] A notification center alerting users about new content, deadlines, and updates.
    \item[$\cdot$ Coding Environment] An embedded coding environment for practice exercises and coding competitions, similar to LeetCode.
    \item[$\cdot$ Error Handling] Consistent error messages displayed in case of invalid input, failed login attempts, or system errors.
    \item[$\cdot$ Keyboard Shortcuts] Common keyboard shortcuts for navigation and execution of commands, enhancing efficiency.
    \item[$\cdot$ Mobile Compatibility:] A responsive design ensuring accessibility on desktops, tablets, and mobile devices.
\end{description}

\subsection{Hardware Interfaces}
$<$Describe the logical and physical characteristics of each interface between 
the software product and the hardware components of the system. This may include 
the supported device types, the nature of the data and control interactions 
between the software and the hardware, and communication protocols to be 
used.$>$

The platform is designed to support various hardware components. Key considerations include:
\begin{description}
    \item[$\cdot$ User Devices:] The platform will be compatible with desktops, laptops, tablets, and smartphones supporting modern web browsers.
    \item[$\cdot$ Server Infrastructure:] Hosted on cloud-based servers (AWS, Google Cloud, or Azure) with auto-scaling to accommodate user load.
    \item[$\cdot$ Peripheral Support:] Users can interact using keyboards, mice, touchscreens, and audio devices for accessibility.
    \item[$\cdot$ Network Requirements:] Requires a stable internet connection with minimum bandwidth to support video streaming and coding environment interaction.
\end{description}

\subsection{Software Interfaces}
$<$Describe the connections between this product and other specific software 
components (name and version), including databases, operating systems, tools, 
libraries, and integrated commercial components. Identify the data items or 
messages coming into the system and going out and describe the purpose of each.  
Describe the services needed and the nature of communications. Refer to 
documents that describe detailed application programming interface protocols.  
Identify data that will be shared across software components. If the data 
sharing mechanism must be implemented in a specific way (for example, use of a 
global data area in a multitasking operating system), specify this as an 
implementation constraint.$>$

The system will integrate with various software components to ensure smooth operation and functionality. These include:
\begin{description}
    \item[$\cdot$ Operating Systems:] Compatible with Windows, macOS, Linux, Android, and iOS.
    \item[$\cdot$ Web Browsers:] Supports the latest versions of Google Chrome, Mozilla Firefox, Safari, and Microsoft Edge.
    \item[$\cdot$ Database Management System:] Uses PostgreSQL or MySQL to store user data, course materials, progress tracking, and assessment records.
    \item[$\cdot$ Authentication Services:] Integration with OAuth2 for third-party authentication (Google, GitHub, etc.) and two-factor authentication (2FA).
    \item[$\cdot$ APIs:] RESTful APIs to connect frontend and backend services, including: User authentication and management, Course content retrieval and management, Assessment grading and tracking, Notification and messaging services, Code execution API for online coding exercises
    \item[$\cdot$ Content Delivery Networks (CDN):] Utilized for efficient media delivery and reduced latency.
    \item[$\cdot$ Notification Services:] Integration with email and push notification services for alerts and reminders.
    \item[$\cdot$ Logging and Monitoring:] Implementation of centralized logging and monitoring tools for system health tracking.
\end{description}


\section{Functional Requirements}
$<$Functional requirements capture the intended behavior of the system. This 
behavior may be expressed as services, tasks or functions the system is required 
to perform. This section is the direct continuation of section 2.2 where you have 
specified the general functional requirements. Here, you should list in detail the
different product functions. $>$

Functional requirements capture the intended behavior of the system. This behavior may be expressed as services, tasks, or functions the system is required to perform. Below is a detailed list of product functions:

\begin{description}
    \item[$\cdot$ User Management:]
        \begin{enumerate}
            \item Users can sign up, log in, and update their profiles.
            \item Role-based access control (Student, Educator, Admin).
            \item Password recovery and security settings.
        \end{enumerate}
    \item[$\cdot$ Course Management:]
        \begin{enumerate}
            \item Educators can create, update, and delete courses.
            \item Upload and manage course materials (videos, PDFs, quizzes).
            \item Students can enroll, drop, and track their progress.
        \end{enumerate}
    \item[$\cdot$ Quiz and Assessment:]
        \begin{enumerate}
            \item MC questions with self-checking and automatic grading.
            \item Short-answer questions (some autograded, some educator-reviewed).
            \item Long-answer questions assigned to educators for grading.
            \item Popup MC questions during the course for engagement.
        \end{enumerate}
    \item[$\cdot$ Progress Tracking:]
        \begin{enumerate}
            \item Roadmap displaying completed and pending modules.
            \item Percentage-based completion tracker.
            \item Assessment performance tracking.
        \end{enumerate}
    \item[$\cdot$ Notifications:]
        \begin{enumerate} 
            \item Automated alerts for deadlines, new content, and assessments.
            \item Customizable notification preferences.
        \end{enumerate}
    \item[$\cdot$ Search and Filter:] Users can search and filter courses, discussions, and assessments.
    \item[$\cdot$ Calendar Integration:] Displays course schedules, assignment deadlines, and learning reminders.
    \item[$\cdot$ Discussion Forum:] Users can ask questions, respond, and interact with educators.
    \item[$\cdot$ Daily Assessment \& Learning Reminders:] Personalized daily learning tasks and notifications.
    \item[$\cdot$ Learning Path Review \& Recommendations:] System-generated personalized course suggestions based on progress.
    \item[$\cdot$ Online Coding Judge:] Users can solve coding problems with real-time execution and have coding competitions with leaderboard tracking.
\end{description}

\section{Use Case Model}
$<$This section is the direct continuation of section 2.3 where you have specified 
the general use cases. Here, you should list in detail the different use cases. 
$>$
\subsection{Use Case \#1}
$<$Describe the use case in detail.$>$

\subsection{Use Case \#2}
$<$Describe the use case in detail.$>$



\chapter{System Features}
$<$This template illustrates organizing the functional requirements for the 
product by system features, the major services provided by the product. You may 
prefer to organize this section by use case, mode of operation, user class, 
object class, functional hierarchy, or combinations of these, whatever makes the 
most logical sense for your product.$>$

\section{System Feature 1}
$<$Don't really say "System Feature 1." State the feature name in just a few 
words.$>$

\subsection{Description and Priority}
$<$Provide a short description of the feature and indicate whether it is of 
High, Medium, or Low priority. You could also include specific priority 
component ratings, such as benefit, penalty, cost, and risk (each rated on a 
relative scale from a low of 1 to a high of 9).$>$

\subsection{Stimulus/Response Sequences}
$<$List the sequences of user actions and system responses that stimulate the 
behavior defined for this feature. These will correspond to the dialog elements 
associated with use cases.$>$

\subsection{Functional Requirements}
$<$Itemize the detailed functional requirements associated with this feature.  
These are the software capabilities that must be present in order for the user 
to carry out the services provided by the feature, or to execute the use case.  
Include how the product should respond to anticipated error conditions or 
invalid inputs. Requirements should be concise, complete, unambiguous, 
verifiable, and necessary. Use "TBD" as a placeholder to indicate when necessary 
information is not yet available.$>$

$<$Each requirement should be uniquely identified with a sequence number or a 
meaningful tag of some kind.$>$

REQ-1:  REQ-2:

\section{System Feature 2 (and so on)}


\chapter{Other Nonfunctional Requirements}

\section{Performance Requirements}
\begin{itemize}
    \item \textbf{Response Time:}
    \begin{itemize}
        \item The platform must load within \textbf{2 seconds} for 95\% of users under normal load conditions.
        \item Code execution results must be returned within \textbf{5 seconds} for 99\% of submissions, even during peak usage.
        \item Real-time collaboration features (if any) must have a latency of less than \textbf{200ms} for seamless interaction.
    \end{itemize}
    
    \item \textbf{Scalability:}
    \begin{itemize}
        \item The platform must support up to \textbf{10,000 concurrent users} without degradation in performance.
    \end{itemize}
    
    \item \textbf{Availability:}
    \begin{itemize}
        \item The platform must have an uptime of \textbf{99.9\%} (excluding scheduled maintenance).
        \item Downtime for scheduled maintenance must not exceed \textbf{10 hours per month} and must be communicated to users at least \textbf{24 hours} in advance.
    \end{itemize}

    \item \textbf{Resource Usage:}
    \begin{itemize}
        \item Code execution environments must not exceed \textbf{10MB of memory} per user session.
        \item The platform must handle \textbf{10,000 code submissions per minute} without performance degradation.
    \end{itemize}
\end{itemize}

\section{Safety Requirements}
\begin{itemize}
    \item \textbf{User Safety:}
    \begin{itemize}
        \item The platform must prevent users from executing malicious code that could harm the system.
        \item A sandboxed environment must be used for code execution.
    \end{itemize}

    \item \textbf{Data Safety:}
    \begin{itemize}
        \item Regular backups of user data must be performed daily and stored securely for at least \textbf{30 days}.
        \item In case of data loss, user data must be restorable to a state no older than \textbf{24 hours}.
    \end{itemize}

    \item \textbf{Compliance:}
    \begin{itemize}
        \item The platform must comply with \textbf{GDPR} and \textbf{COPPA}.
    \end{itemize}
\end{itemize}

\section{Security Requirements}
\begin{itemize}
    \item \textbf{Authentication and Authorization:}
    \begin{itemize}
        \item Users must authenticate via \textbf{OAuth 2.0} or email/password with \textbf{MFA} support.
        \item Role-based access control (RBAC) must be implemented.
    \end{itemize}

    \item \textbf{Data Security:}
    \begin{itemize}
        \item All sensitive data must be encrypted using \textbf{AES-256}.
        \item All data in transit must be secured using \textbf{TLS 1.2} or higher.
    \end{itemize}

    \item \textbf{Prevention of Abuse:}
    \begin{itemize}
        \item The platform must prevent brute-force attacks, SQL injection, and XSS attacks.
        \item Rate limiting must be implemented (e.g., max \textbf{10 submissions per minute} per user).
    \end{itemize}

    \item \textbf{Compliance:}
    \begin{itemize}
        \item The platform must comply with \textbf{ISO/IEC 27001} and \textbf{SOC 2 Type II}.
    \end{itemize}
\end{itemize}

\section{Software Quality Attributes}
\begin{itemize}
    \item \textbf{Usability:}
    \begin{itemize}
        \item The platform must have an intuitive UI with a \textbf{90\% satisfaction rate}.
        \item The learning curve for new users should not exceed \textbf{10 minutes}.
    \end{itemize}

    \item \textbf{Reliability:}
    \begin{itemize}
        \item The platform must have \textbf{99.9\% uptime} and recover from failures within \textbf{5 minutes}.
    \end{itemize}

    \item \textbf{Maintainability:}
    \begin{itemize}
        \item The codebase must follow modular design principles, with at least \textbf{80\% test coverage}.
        \item The platform must support seamless updates with zero downtime.
    \end{itemize}

    \item \textbf{Portability:}
    \begin{itemize}
        \item The platform must be compatible with major browsers and OSs.
        \item Mobile responsiveness must be ensured for screen sizes as small as \textbf{320px}.
    \end{itemize}

    \item \textbf{Performance:}
    \begin{itemize}
        \item The platform must support \textbf{10,000 concurrent users} with a response time of less than \textbf{2 seconds}.
    \end{itemize}
\end{itemize}

\section{Business Rules}
\begin{itemize}
    \item \textbf{User Roles and Permissions:}
    \begin{itemize}
        \item \textbf{Admins}: Manage challenges, accounts, and analytics.
        \item \textbf{Premium Users}: Access advanced features.
        \item \textbf{Users}: Access basic challenges and competitions.
    \end{itemize}

    \item \textbf{Content Moderation:}
    \begin{itemize}
        \item User-submitted content must be reviewed for inappropriate content.
        \item Users violating the code of conduct must be warned or banned.
    \end{itemize}

    \item \textbf{Monetization:}
    \begin{itemize}
        \item Premium features must be available via a subscription model.
        \item Free users must be limited to \textbf{10 challenges per day}.
    \end{itemize}

    \item \textbf{Competitions and Rankings:}
    \begin{itemize}
        \item Coding competitions must occur at least \textbf{once a month}.
        \item User rankings must be updated in real-time.
    \end{itemize}

    \item \textbf{Data Retention:}
    \begin{itemize}
        \item User data must be retained for at least \textbf{6 months} after account deletion.
        \item Inactive accounts (no login for \textbf{1 year}) must be archived and deleted after \textbf{3 months} of notification.
    \end{itemize}
\end{itemize}


\chapter{Other Requirements}
$<$Define any other requirements not covered elsewhere in the SRS. This might 
include database requirements, internationalization requirements, legal 
requirements, reuse objectives for the project, and so on. Add any new sections 
that are pertinent to the project.$>$

This section defines additional requirements that are not covered elsewhere in the document.

\begin{description}
    \item[$\cdot$ Database Requirements:] The system should support relational databases with ACID compliance.
    \item[$\cdot$ Internationalization:] The platform should support multiple languages, including English, Spanish, and Chinese.
    \item[$\cdot$ Legal Compliance:] The platform must adhere to data protection regulations (e.g., GDPR, CCPA) and accessibility laws.
    \item[$\cdot$ Reuse Objectives:] Modular architecture should be adopted to allow for code reuse across projects and facilitate easier maintenance.
    \item[$\cdot$ Scalability Requirements:] The system should be designed to handle high concurrent users efficiently.
    \item[$\cdot$ Backup and Recovery:] Automated daily backups must be implemented, with a recovery plan in place for data loss scenarios.
    \item[$\cdot$ Customization and Extensibility:] The system should allow customization for branding and the addition of new features without major rework.
\end{description}

\addchap{Appendix A: Glossary}
%see https://en.wikibooks.org/wiki/LaTeX/Glossary
$<$Define all the terms necessary to properly interpret the SRS, including 
acronyms and abbreviations. You may wish to build a separate glossary that spans 
multiple projects or the entire organization, and just include terms specific to 
a single project in each SRS.$>$

\addchap{Appendix B: Analysis Models}
$<$Optionally, include any pertinent analysis models, such as data flow 
diagrams, class diagrams, state-transition diagrams, or entity-relationship 
diagrams.$>$

\addchap{Appendix C: To Be Determined List}
$<$Collect a numbered list of the TBD (to be determined) references that remain 
in the SRS so they can be tracked to closure.$>$

\end{document}
