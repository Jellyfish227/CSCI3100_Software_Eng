\def\myversion{1.0}
\def\projectname{Kaiju Academy}

\documentclass[a4paper, 11pt]{scrreprt}
\usepackage{srs}  % Load SRS document style

% Document-specific settings
\hypersetup{
    pdftitle={Software Requirement Specification},
    pdfauthor={C. H. Yu},
    pdfsubject={Technical Documentation},
    pdfkeywords={LaTeX, SRS, Software Requirements Specification, Software}
}

% Reduce itemize spacing
\usepackage{enumitem}
\setlist[itemize]{noitemsep,topsep=0pt,parsep=0pt,partopsep=0pt}

\begin{document}

\pagenumbering{roman}  % Start with roman numerals for front matter

\begin{titlepage}
    \begin{flushright}
        \rule{\textwidth}{5pt}\vskip1cm
        \begin{bfseries}
            \Huge{SOFTWARE REQUIREMENTS\\ SPECIFICATION}\\
            \vspace{1.6cm}
            for\\
            \vspace{1.6cm}
            \projectname\\  % Use project name variable
            \vspace{1.6cm}
            \LARGE{Version \myversion}\\
            \vspace{1.6cm}
            Prepared by\\
            Group A2\\
            \Large{
                \begin{tabularx}{\textwidth}{l l >{\raggedleft\arraybackslash}X}
                    YU Ching Hei & 1155193237 & \href{mailto:chyu@link.cuhk.edu.hk}{chyu@link.cuhk.edu.hk}\\
                    Lei Hei Tung & 1155194969 & \href{mailto:1155194969@link.cuhk.edu.hk}{1155194969@link.cuhk.edu.hk}\\
                    Ankhbayar Enkhtaivan & 1155185142 & \href{mailto:1155185142@link.cuhk.edu.hk}{1155185142@link.cuhk.edu.hk}\\
                    Yum Ho Kan & 1155195234 & \href{mailto:1155195234@link.cuhk.edu.hk}{1155195234@link.cuhk.edu.hk}\\
                    Leung Chung Wang & 1155194650 & \href{mailto:1155194650@link.cuhk.edu.hk}{1155194650@link.cuhk.edu.hk}\\
                \end{tabularx}
            }\\
            \vspace{1.6cm}
            The Chinese University of Hong Kong\\
            Department of Computer Science and Engineering\\
            CSCI3100: Software Engineering\\
            \vspace{1.6cm}
            \today\\
        \end{bfseries}
    \end{flushright}
\end{titlepage}

\setuptoc{toc}{totoc}  % Add ToC to itself using KOMA-Script
\tableofcontents

\addchap{Document Revision History}  % Unnumbered chapter that appears in ToC

\begin{center}
    \begin{tabularx}{\textwidth}{>{\raggedright\arraybackslash}p{2cm}>{\raggedright\arraybackslash}p{3cm}>{\raggedright\arraybackslash}p{3cm}>{\raggedright\arraybackslash}X}
        \toprule
        Version & Revised By & Revision Date & Comments\\
        \midrule
        0.1 & C. H. Yu & 2025-02-08 & \begin{revisionitem}[Updated:]
            \item Initial document structure
            \item Basic template setup
        \end{revisionitem}\\
        \midrule
        0.2 & C. H. Yu & 2025-02-08 & \begin{revisionitem}[Updated:]
            \item Formatting
        \end{revisionitem}\\
        \midrule
        0.3 & C. H. Yu & 2025-02-08 & \begin{revisionitem}[Updated:]
            \item Titlepage formatting
            \item Page numbering
            \item Chapters title formatting
            \item Chapter and sections arrangement
        \end{revisionitem}\\
        \midrule
        0.4 & H. T. Lei & 2025-02-09 & \begin{revisionitem}[Added:]
            \item Specific requirements
        \end{revisionitem}\\
        \midrule
        0.4.1 & C. H. Yu & 2025-02-10 & \begin{revisionitem}[Updated:]
            \item Specific requirements: fixed compilation error
            \item Titlepage formatting: fixed alignment issue
        \end{revisionitem}\\
        \midrule
        0.5 & C. H. Yu & 2025-02-10 & \begin{revisionitem}[Added:]
            \item Acronyms and Abbreviations table
            \item Product Overview
            \item Product Functionality
            \item User Classes and Characteristics
            \item Design and Implementation Constraints
            \item Assumptions and Dependencies
        \end{revisionitem}\\
        \midrule
        0.6 & A. Enkhtaivan & 2025-02-10 & \begin{revisionitem}[Updated:]
            \item System Features
            \item Specific requirements Use Case Model
        \end{revisionitem}\\
        \midrule
        0.7 & H. K. Yum & 2025-02-10 & \begin{revisionitem}[Updated:]
            \item Nonfunctional Requirements
        \end{revisionitem}\\
        \midrule
        0.8 & C. W. Leung & 2025-02-10 & \begin{revisionitem}[Added:]
            \item Document Purpose
            \item Project Scope
            \item Intended Audience and Document Overview
            \item Document Conventions
            \item References and Acknowledgments
        \end{revisionitem}\\
        \midrule
        0.9 & H. T. Lei & 2025-02-10 & \begin{revisionitem}[Updated:]
            \item Refine Nonfunctional Requirements
        \end{revisionitem}\\
        \midrule
        1.0 & \begin{tabular}[t]{@{}l@{}}C. H. Yu\\ H. T. Lei\\ A. Enkhtaivan\\ H. K. Yum\\ C. W. Leung\end{tabular} & 2025-02-10 & \begin{revisionitem}[Updated:]
            \item Finalized draft with all requirements completed and reviewed
        \end{revisionitem}\\
        \bottomrule

    \end{tabularx}
\end{center}

\clearpage
\pagenumbering{arabic}  % Switch to arabic numbers for main content

\chapter{Introduction}
\section{Document Purpose}
This document encompasses all essential functionalities,
including course delivery, user progress tracking, community interaction, and assessment tools, providing both
high-level and specific requirements. This document shall form the basis for all stakeholders and developers to
understand intended features and constraints of the platform.

\section{Project Scope}
Kaiju Academy is an online self-learning platform that makes learning to code easier and more engaging. The platform delivers interactive coding courses for all skill levels through a self-paced learning environment with progress tracking and instant feedback. Game-like elements drive student engagement, while community features enable collaboration between learners and educators. Through this technology-driven learning experience, Kaiju Academy aims to expand digital education access and develop a skilled tech workforce.

\section{Intended Audience and Document Overview}
This document serves as a guide for developers, managers, testers and stakeholders of Kaiju Academy. It contains sections covering product description, requirements, features, and non-functional requirements. We recommend starting with the product description section before moving to specific requirements.

\section{Definitions, Acronyms and Abbreviations}
\begin{center}
    \begin{tabularx}{\textwidth}{lX}
        \toprule
        \textbf{Abbreviation} & \textbf{Definition} \\
        \midrule
        2FA & Two-factor Authentication \\
        API & Application Programming Interface \\
        CDN & Content Delivery Networks \\
        GDPR & General Data Protection Regulation \\
        LMS & Learning Management System \\
        MC & Multiple Choice \\
        REST & Representational State Transfer \\
        SRS & Software Requirements Specification \\
        UI & User Interface \\
        UX & User Experience \\
        \bottomrule
    \end{tabularx}
\end{center}


\section{Document Conventions}
This Project Requirements Specification adheres to the IEEE Std 830-1998 standard. The document utilizes Times New Roman font in size 11 for consistency and readability. Important terms are highlighted in bold, while supplementary notes are presented in italics. Each requirement is uniquely identified, with higher-level requirements inheriting their priority unless otherwise specified. Sections and requirements are sequentially numbered to facilitate easy navigation. Additionally, technical terms and acronyms are clearly defined in the Glossary to ensure clarity and understanding for all readers.

\section{References and Acknowledgments}
IEEE. IEEE Std 830-1998 IEEE Recommended Practice for Software Requirements Specifications. IEEE Computer Society, 1998.


\chapter{Overall Description}
\section{Product Overview}
% $<$Describe the context and origin of the product being specified in this SRS.  
% For example, state whether this product is a follow-on member of a product 
% family, a replacement for certain existing systems, or a new, self-contained 
% product. If the SRS defines a component of a larger system, relate the 
% requirements of the larger system to the functionality of this software and 
% identify interfaces between the two. A simple diagram that shows the major 
% components of the overall system, subsystem interconnections, and external 
% interfaces can be helpful.$>$

Kaiju Academy is a web-based e-learning platform that provides interactive programming education through modern LMS capabilities and specialized coding features, facilitating the learning of programming anytime and anywhere as long as you are connected to the internet.\\

The platform serves as a comprehensive learning environment where students can engage with programming concepts through a structured curriculum. Users access the platform through web browsers, eliminating the need for software installation reducing the barrier to entry for users. \\

The platform employs a module-based learning approach, where content is organized into discrete units that build upon each other. Each module typically contains:

\begin{itemize}
    \item Video lectures with synchronized transcripts
    \item Interactive coding exercises
    \item Supplementary reading materials
    \item Practice problems and assignments
    \item Progress assessments
\end{itemize}

Students can track their progress through a personalized dashboard, which displays completed modules, upcoming assignments, and achievement metrics. The integrated code editor allows students to practice coding directly within the browser, with immediate feedback and automated grading capabilities. \\

Educators can utilize the platform to create courses, manage course materials and provide targeted assistance where needed. They could also answer questions from students in the discussion forum. The system's analytics tools help identify areas where students may be struggling, enabling educators to provide additional teaching resources. The platform features automated grading and feedback for assignments, allowing educators to have more flexible arrangement. 

\section{Product Functionality}

Kaiju Academy provides the following major functionalities:

\begin{itemize}
    \item User Management and Authentication
    \item Course Creation and Management
    \item Interactive Learning Environment
    \item Assessment and Progress Tracking
    \item Community and Discussion Features
    \item Administrative Tools
\end{itemize}

\section{User Classes and Characteristics}

Kaiju Academy serves four primary user classes:

\begin{description}
    \item[Administrators:] System managers with full platform access for maintenance and monitoring
    \item[Students:] Main users who access courses and track their learning progress
    \item[Educators:] Subject matter experts who create and manage course content
    \item[Content Moderators:] Forum managers who maintain community standards
\end{description}

\section{Design and Implementation Constraints}

The implementation of Kaiju Academy is governed by the following key constraints:

\begin{description}
    \item[Hardware Constraints]\mbox{}
        \begin{itemize}
            \item We should able to run the platform with low memory usage
        \end{itemize}
    
    \item[Performance Constraints]\mbox{}
        \begin{itemize}
            \item Page load time must not exceed 3 seconds
            \item System must support at least 1000 concurrent users
            \item Video streaming must adapt to user bandwidth
            \item Code execution response time must be under 5 seconds
            \item Database queries must complete within 1 second
        \end{itemize}
    
    \item[Operational Constraints]\mbox{}
        \begin{itemize}
            \item System must achieve 99.9\% uptime
            \item Must implement automated backup systems
            \item Must support horizontal scaling
            \item Must implement monitoring and logging
            \item Must provide disaster recovery procedures
        \end{itemize}
\end{description}

\section{Assumptions and Dependencies}

\subsection{Assumptions}
\begin{itemize}
    \item \textbf{Internet access:} Since the application relies on a web interface, users must have stable internet connectivity with minimum 5 Mbps bandwidth or access through institution's local network
    
    \item \textbf{Minimum System Requirements:} All client devices must feature:
        \begin{itemize}
            \item Computer or mobile device with minimum 4GB RAM for smooth performance
            \item Web browser (Chrome, Firefox, Safari, Edge - latest 2 versions)
            \item HTML5 and JavaScript enabled
        \end{itemize}
\end{itemize}

\subsection{Dependencies}
\begin{itemize}
    \item \textbf{Third-Party Libraries and Services:}
        \begin{itemize}
            \item Authentication protocols (OAuth2)
            \item Database management systems (PostgreSQL, MongoDB, Redis)
            \item Cloud infrastructure services (AWS/Google Cloud/DigitalOcean)
            \item Content delivery networks (CDN)
            \item Video hosting platforms
            \item Code editor components (CodeMirror/Monaco)
        \end{itemize}
    
    \item \textbf{Development Methodologies:}
        \begin{itemize}
            \item UML Modeling: System architecture documentation and design specifications
            \item COMET Methodology: Concurrent object modeling for workflow integration
            \item Container orchestration with Docker and Kubernetes
            \item CI/CD pipeline implementation
        \end{itemize}
\end{itemize}

\subsection{Critical Risks}
\begin{itemize}
    \item \textbf{External System Failures:}
        \begin{itemize}
            \item Third-party service disruptions (authentication, video hosting)
            \item Cloud infrastructure outages
            \item CDN performance issues
            \item Database system failures
        \end{itemize}
\end{itemize}


\chapter{Specific Requirements}

\section{External Interface Requirements}
\subsection{User Interfaces}

Kaiju Academy will have a graphical user interface (GUI) designed for ease of use and accessibility across different roles, including students, educators, and administrators. The UI components include:
\begin{description}
    \item[$\cdot$ Dashboard:] Upon login, users will be presented with a dashboard displaying their enrolled courses, upcoming deadlines, progress tracking, and notifications.
    \item[$\cdot$ Navigation:] A global navigation bar providing quick access to Courses, Assessments, Calendar, Discussion Forum, and Profile settings.
    \item[$\cdot$ Course Pages:] Each course contains a structured layout displaying modules, videos, PDFs, quizzes, and assessments with a progress tracker.
    \item[$\cdot$ Assessment Interface:] Interactive assessment screens allowing multiple-choice (MC) questions with self-checking, short-answer autograded questions, and long-answer questions submitted for manual grading.
    \item[$\cdot$ Progress Tracking:] A visual roadmap for students to track completed and pending modules and overall course progress.
    \item[$\cdot$ Discussion Forum:] A interactive discussion forum with search, filter, and sort options.
    \item[$\cdot$ Calendar:] An integrated calendar highlighting course schedules, assignment deadlines, and upcoming events.
    \item[$\cdot$ Notifications:] A notification center alerting users about new content, deadlines, and updates.
    \item[$\cdot$ Coding Environment] An embedded coding environment for practice exercises and coding competitions, similar to LeetCode.
    \item[$\cdot$ Error Handling] Consistent error messages displayed in case of invalid input, failed login attempts, or system errors.
    \item[$\cdot$ Keyboard Shortcuts] Common keyboard shortcuts for navigation and execution of commands, enhancing efficiency.
    \item[$\cdot$ Mobile Compatibility:] A responsive design ensuring accessibility on desktops, tablets, and mobile devices.
\end{description}

\subsection{Hardware Interfaces}

The platform is designed to support various hardware components. Key considerations include:
\begin{description}
    \item[$\cdot$ User Devices:] The platform will be compatible with desktops, laptops, tablets, and smartphones supporting modern web browsers.
    \item[$\cdot$ Server Infrastructure:] Hosted on cloud-based servers (AWS, Google Cloud, or Azure) with auto-scaling to accommodate user load.
    \item[$\cdot$ Peripheral Support:] Users can interact using keyboards, mice, touchscreens, and audio devices for accessibility.
    \item[$\cdot$ Network Requirements:] Requires a stable internet connection with minimum bandwidth to support video streaming and coding environment interaction.
\end{description}

\subsection{Software Interfaces}

The system will integrate with various software components to ensure smooth operation and functionality. These include:
\begin{description}
    \item[$\cdot$ Operating Systems:] Compatible with Windows, macOS, Linux, Android, and iOS.
    \item[$\cdot$ Web Browsers:] Supports the latest versions of Google Chrome, Mozilla Firefox, Safari, and Microsoft Edge.
    \item[$\cdot$ Database Management System:] Uses PostgreSQL or MySQL to store user data, course materials, progress tracking, and assessment records.
    \item[$\cdot$ Authentication Services:] Integration with OAuth2 for third-party authentication (Google, GitHub, etc.) and two-factor authentication (2FA).
    \item[$\cdot$ APIs:] RESTful APIs to connect frontend and backend services, including: User authentication and management, Course content retrieval and management, Assessment grading and tracking, Notification and messaging services, Code execution API for online coding exercises
    \item[$\cdot$ Content Delivery Networks (CDN):] Utilized for efficient media delivery and reduced latency.
    \item[$\cdot$ Notification Services:] Integration with email and push notification services for alerts and reminders.
    \item[$\cdot$ Logging and Monitoring:] Implementation of centralized logging and monitoring tools for system health tracking.
\end{description}


\section{Functional Requirements}

Functional requirements capture the intended behavior of the system. This behavior may be expressed as services, tasks, or functions the system is required to perform. Below is a detailed list of product functions:

\begin{table}[h!]
    \centering
    \begin{tabular}{|c|l|p{10cm}|}
    \hline
    \textbf{Function ID} & \textbf{Function Name} & \textbf{Description} \\
    \hline
    FR01 & User Registration & Users can create an account by providing necessary details. \\
    FR02 & User Login & Users can log in using credentials. \\
    FR03 & Profile Management & Users can update personal details and account settings. \\
    FR04 & Role Management & Assign roles (Student, Educator, Admin) with specific permissions. \\
    FR05 & Course Creation & Educators can create and manage courses. \\
    FR06 & Course Enrollment & Students can enroll or drop courses. \\
    FR07 & Course Content Management & Educators can upload and organize course materials. \\
    FR08 & Quiz Management & Educators can create and assign quizzes. \\
    FR09 & Automated Grading & System auto-grades multiple-choice and short-answer questions. \\
    FR10 & Manual Grading & Educators grade long-answer questions. \\
    FR11 & Progress Tracking & Users can track their learning progress. \\
    FR12 & Notifications & System alerts users of deadlines, updates, and activities. \\
    FR13 & Calendar Integration & Displays schedules, deadlines, and reminders. \\
    FR14 & Discussion Forum & Users can post, reply, and discuss course-related topics. \\
    FR15 & Online Coding Environment & Users can solve coding problems with real-time execution. \\
    FR16 & Code Competitions & Users can participate in coding challenges with leaderboards. \\
    FR17 & Search & Users can search and filter courses, quizzes, and discussions. \\
    FR18 & Learning Recommendations & AI-based suggestions for personalized learning paths. \\
    FR19 & Data Backup & System automatically backs up data periodically. \\
    FR20 & Security & Implements authentication, authorization, and encryption. \\
    \hline
    \end{tabular}
    \caption{Functional Requirements Table}
\end{table}

\section{Use Case Model}
\subsection{Use Case 1: View Course Progress}
\subsubsection{Description}
This use case describes how a student can view their progress in any course

\subsubsection{Actors and Preconditions}
\begin{itemize}
    \item \textbf{Actors}: Student
    \item \textbf{Preconditions}: The student must be logged in and enrolled in the course.
\end{itemize}

\subsubsection{Steps}
\begin{enumerate}
    \item The student navigates to the "Courses" section.
    \item The student selects a course.
    \item The system displays the course progress.
\end{enumerate}

\subsection{Use Case 2: Course Modification}
\subsubsection{Description}
This use case describes the process of an educator modifies a course on the Kaiju Academy platform.

\subsubsection{Actors and Preconditions}
\begin{itemize}
    \item \textbf{Actors}: Educator
    \item \textbf{Preconditions}: The educator must be logged in and have the necessary permissions to modify courses.
\end{itemize}

\subsubsection{Steps}
\begin{enumerate}
    \item The educator navigates to the "Course Management" section.
    \item The educator enters course details (e.g., title, description, modules, and materials).
    \item The system validates the course details.
    \item The system adds the course to the platform and notifies enrolled students (if applicable).
\end{enumerate}

\subsubsection{Alternative Flow}
\begin{itemize}
    \item \textbf{Step 4a}: If the educator provides invalid details (e.g., missing title), the system displays an error and prompts corrections.
    \item \textbf{Step 5a}: If the educator cancels, the system discards the changes and returns to the "Course Management" section.
\end{itemize}

\chapter{System Features}
This sections serves to describe the functional requirements of the product by its major features. The features are described in detail, including their development priority, user interactions, and specific functions.

\section{User Dashboard}


\subsection{Description and Priority}
The User Dashboard is the High Priority feature as the hub is where users can access their courses, track their progress, view upcoming deadlines, and receive notifications.

\subsection{Stimulus/Response Sequences}
\begin{itemize}
    \item \textbf{User Action}: Logs into the system. \\
          \textbf{System Response}: Displays the dashboard with course progress, deadlines, and notifications.
    \item \textbf{User Action}: Clicks on a course. \\
          \textbf{System Response}: Redirects to the course page, displaying modules, videos, and assessments.
    \item \textbf{User Action}: Marks a module as completed. \\
          \textbf{System Response}: Updates the progress tracker and highlights the next module.
\end{itemize}

\subsection{Functional Requirements}
\subsubsection{Functional Requirements}
\begin{itemize}
    \item \textbf{REQ-1}: The dashboard must display enrolled courses, progress, and deadlines upon login.
    \item \textbf{REQ-2}: Users should be able to navigate to course pages directly from the dashboard.
    \item \textbf{REQ-3}: Progress tracking must update in real-time when a module is marked as completed.
\end{itemize}

\subsection{Course Management}
\subsubsection{Description and Priority}
The \textbf{Course Management} feature is the High Priority feature as the feature enables educators to create, update, and delete courses, as well as manage course materials such as videos, PDFs, and quizzes. It ensures that educators can efficiently organize and maintain course content, providing a structured learning experience.

\subsubsection{Stimulus/Response Sequences}
\begin{itemize}
    \item \textbf{Educator Action}: Creates a new course. \\
          \textbf{System Response}: Displays section to the educator to upload course materials and set deadlines.
    \item \textbf{Educator Action}: Updates a course module. \\
          \textbf{System Response}: Saves changes and notifies enrolled students of updates.
    \item \textbf{Educator Action}: Deletes a course. \\
          \textbf{System Response}: Removes the course and all associated materials from the system.
\end{itemize}

\subsubsection{Functional Requirements}
\begin{itemize}
    \item \textbf{REQ-1}: Educators must be able to create, update, and delete courses.
    \item \textbf{REQ-2}: Course materials (videos, PDFs, quizzes) must be uploaded and organized by modules.
    \item \textbf{REQ-3}: Students should receive notifications when course materials are updated.
\end{itemize}

\subsection{Assessment and Grading}
\subsubsection{Description and Priority}
The \textbf{Assessment and Grading} feature allows educators to create quizzes and assessments, while students can complete them and receive grades. This feature is of \textbf{Medium Priority} as it supports the learning process but is secondary to content delivery. It provides a mechanism for evaluating student understanding and progress.

\subsubsection{Stimulus/Response Sequences}
\begin{itemize}
    \item \textbf{Educator Action}: Creates a quiz with multiple-choice (MC) and short-answer questions. \\
          \textbf{System Response}: Saves the quiz and makes it available to students.
    \item \textbf{Student Action}: Completes a quiz. \\
          \textbf{System Response}: Automatically grades MC questions and displays results.
    \item \textbf{Educator Action}: Reviews and grades short-answer questions. \\
          \textbf{System Response}: Updates the student's grade and provides feedback.
\end{itemize}

\subsubsection{Functional Requirements}
    \begin{itemize}
    \item \textbf{REQ-1}: Educators must be able to create quizzes with MC and short-answer questions along with coding assessments with the test cases.
    \item \textbf{REQ-2}: MC, and coding assessment questions must be automatically graded, and results displayed immediately.
    \item \textbf{REQ-3}: Educators should be able to manually and use AI prompt grade short-answer questions and provide feedback.
\end{itemize}

\chapter{Other Nonfunctional Requirements}

\section{Performance Requirements}
\begin{itemize}
    \item \textbf{Response Time:}
    \begin{itemize}
        \item The platform must load within \textbf{2 seconds} for 95\% of users under normal load conditions.
        \item Code execution results must be returned within \textbf{5 seconds} for 99\% of submissions, even during peak usage.
        \item Real-time collaboration features (if any) must have a latency of less than \textbf{200ms} for seamless interaction.
    \end{itemize}

    \item \textbf{Scalability:}
    \begin{itemize}
        \item The platform must support up to \textbf{10,000 concurrent users} without degradation in performance.
    \end{itemize}

    \item \textbf{Availability:}
    \begin{itemize}
        \item The platform must have an uptime of \textbf{99.9\%} (excluding scheduled maintenance).
        \item Downtime for scheduled maintenance must not exceed \textbf{10 hours per month} and must be communicated to users at least \textbf{24 hours} in advance.
    \end{itemize}

    \item \textbf{Resource Usage:}
    \begin{itemize}
        \item Code execution environments must not exceed \textbf{10MB of memory} per user session.
        \item The platform must handle \textbf{10,000 code submissions per minute} without performance degradation.
    \end{itemize}
\end{itemize}

\section{Safety Requirements}
\begin{itemize}
    \item \textbf{User Safety:}
    \begin{itemize}
        \item The platform must prevent users from executing malicious code that could harm the system.
        \item A sandboxed environment must be used for code execution.
    \end{itemize}

    \item \textbf{Data Safety:}
    \begin{itemize}
        \item Regular backups of user data must be performed daily and stored securely for at least \textbf{30 days}.
        \item In case of data loss, user data must be restorable to a state no older than \textbf{24 hours}.
    \end{itemize}

    \item \textbf{Compliance:}
    \begin{itemize}
        \item The platform must comply with \textbf{GDPR} and \textbf{COPPA}.
    \end{itemize}
\end{itemize}

\section{Security Requirements}
\begin{itemize}
    \item \textbf{Authentication and Authorization:}
    \begin{itemize}
        \item Users must authenticate via \textbf{OAuth 2.0} or email/password with \textbf{MFA} support.
        \item Role-based access control (RBAC) must be implemented.
    \end{itemize}

    \item \textbf{Data Security:}
    \begin{itemize}
        \item All sensitive data must be encrypted using \textbf{AES-256}.
        \item All data in transit must be secured using \textbf{TLS 1.2} or higher.
    \end{itemize}

    \item \textbf{Prevention of Abuse:}
    \begin{itemize}
        \item The platform must prevent brute-force attacks, SQL injection, and XSS attacks.
        \item Rate limiting must be implemented (e.g., max \textbf{10 submissions per minute} per user).
    \end{itemize}

    \item \textbf{Compliance:}
    \begin{itemize}
        \item The platform must comply with \textbf{ISO/IEC 27001} and \textbf{SOC 2 Type II}.
    \end{itemize}
\end{itemize}

\section{Software Quality Attributes}
\begin{itemize}
    \item \textbf{Usability:}
    \begin{itemize}
        \item The platform must have an intuitive UI with a \textbf{90\% satisfaction rate}.
        \item The learning curve for new users should not exceed \textbf{10 minutes}.
    \end{itemize}

    \item \textbf{Reliability:}
    \begin{itemize}
        \item The platform must have \textbf{99.9\% uptime} and recover from failures within \textbf{5 minutes}.
    \end{itemize}

    \item \textbf{Maintainability:}
    \begin{itemize}
        \item The codebase must follow modular design principles, with at least \textbf{80\% test coverage}.
        \item The platform must support seamless updates with zero downtime.
    \end{itemize}

    \item \textbf{Portability:}
    \begin{itemize}
        \item The platform must be compatible with major browsers and OSs.
        \item Mobile responsiveness must be ensured for screen sizes as small as \textbf{320px}.
    \end{itemize}

    \item \textbf{Performance:}
    \begin{itemize}
        \item The platform must support \textbf{10,000 concurrent users} with a response time of less than \textbf{2 seconds}.
    \end{itemize}
\end{itemize}

\section{Additional Requirements}
This section defines additional requirements that are not covered elsewhere in the document.

\begin{description}
    \item[$\cdot$ Database Requirements:] The system should support relational databases with ACID compliance.
    \item[$\cdot$ Internationalization:] The platform should support multiple languages, including English, Spanish, and Chinese.
    \item[$\cdot$ Legal Compliance:] The platform must adhere to data protection regulations (e.g., GDPR, CCPA) and accessibility laws.
    \item[$\cdot$ Reuse Objectives:] Modular architecture should be adopted to allow for code reuse across projects and facilitate easier maintenance.
    \item[$\cdot$ Scalability Requirements:] The system should be designed to handle high concurrent users efficiently.
    \item[$\cdot$ Backup and Recovery:] Automated daily backups must be implemented, with a recovery plan in place for data loss scenarios.
    \item[$\cdot$ Customization and Extensibility:] The system should allow customization for branding and the addition of new features without major rework.
\end{description}

\end{document}