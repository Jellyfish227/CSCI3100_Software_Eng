\def\myversion{1.1}
\def\projectname{Kaiju Academy}

\documentclass[a4paper, 11pt]{scrreprt}
\usepackage{srs}  % Load SRS document style

% Document-specific settings
\hypersetup{
    pdftitle={Software Requirement Specification},
    pdfauthor={C. H. Yu},
    pdfsubject={Technical Documentation},
    pdfkeywords={LaTeX, SRS, Software Requirements Specification, Software}
}

% Reduce itemize spacing
\usepackage{enumitem}
\setlist[itemize]{noitemsep,topsep=0pt,parsep=0pt,partopsep=0pt}

\begin{document}

\pagenumbering{roman}  % Start with roman numerals for front matter

\begin{titlepage}
    \begin{flushright}
        \rule{\textwidth}{5pt}\vskip1cm
        \begin{bfseries}
            \Huge{SOFTWARE REQUIREMENTS\\ SPECIFICATION}\\
            \vspace{1.6cm}
            for\\
            \vspace{1.6cm}
            \projectname\\  % Use project name variable
            \vspace{1.6cm}
            \LARGE{Version \myversion}\\
            \vspace{1.6cm}
            Prepared by\\
            Group A2\\
            \Large{
                \begin{tabularx}{\textwidth}{l l >{\raggedleft\arraybackslash}X}
                    YU Ching Hei & 1155193237 & \href{mailto:chyu@link.cuhk.edu.hk}{chyu@link.cuhk.edu.hk}\\
                    Lei Hei Tung & 1155194969 & \href{mailto:1155194969@link.cuhk.edu.hk}{1155194969@link.cuhk.edu.hk}\\
                    Ankhbayar Enkhtaivan & 1155185142 & \href{mailto:1155185142@link.cuhk.edu.hk}{1155185142@link.cuhk.edu.hk}\\
                    Yum Ho Kan & 1155195234 & \href{mailto:1155195234@link.cuhk.edu.hk}{1155195234@link.cuhk.edu.hk}\\
                    Leung Chung Wang & 1155194650 & \href{mailto:1155194650@link.cuhk.edu.hk}{1155194650@link.cuhk.edu.hk}\\
                \end{tabularx}
            }\\
            \vspace{1.6cm}
            The Chinese University of Hong Kong\\
            Department of Computer Science and Engineering\\
            CSCI3100: Software Engineering\\
            \vspace{1.6cm}
            \today\\
        \end{bfseries}
    \end{flushright}
\end{titlepage}

\setuptoc{toc}{totoc}  % Add ToC to itself using KOMA-Script
\tableofcontents

\addchap{Document Revision History}  % Unnumbered chapter that appears in ToC

\begin{center}
    \begin{tabularx}{\textwidth}{>{\raggedright\arraybackslash}p{2cm}>{\raggedright\arraybackslash}p{3cm}>{\raggedright\arraybackslash}p{3cm}>{\raggedright\arraybackslash}X}
        \toprule
        Version & Revised By & Revision Date & Comments\\
        \midrule
        0.1 & C. H. Yu & 2025-02-08 & \begin{revisionitem}[Updated:]
            \item Initial document structure
            \item Basic template setup
        \end{revisionitem}\\
        \midrule
        0.2 & C. H. Yu & 2025-02-08 & \begin{revisionitem}[Updated:]
            \item Formatting
        \end{revisionitem}\\
        \midrule
        0.3 & C. H. Yu & 2025-02-08 & \begin{revisionitem}[Updated:]
            \item Titlepage formatting
            \item Page numbering
            \item Chapters title formatting
            \item Chapter and sections arrangement
        \end{revisionitem}\\
        \midrule
        0.4 & H. T. Lei & 2025-02-09 & \begin{revisionitem}[Added:]
            \item Specific requirements
        \end{revisionitem}\\
        \midrule
        0.4.1 & C. H. Yu & 2025-02-10 & \begin{revisionitem}[Updated:]
            \item Specific requirements: fixed compilation error
            \item Titlepage formatting: fixed alignment issue
        \end{revisionitem}\\
        \midrule
        0.5 & C. H. Yu & 2025-02-10 & \begin{revisionitem}[Added:]
            \item Acronyms and Abbreviations table
            \item Product Overview
            \item Product Functionality
            \item User Classes and Characteristics
            \item Design and Implementation Constraints
            \item Assumptions and Dependencies
        \end{revisionitem}\\
        \midrule
        0.6 & A. Enkhtaivan & 2025-02-10 & \begin{revisionitem}[Updated:]
            \item System Features
            \item Specific requirements Use Case Model
        \end{revisionitem}\\
        \midrule
        0.7 & H. K. Yum & 2025-02-10 & \begin{revisionitem}[Updated:]
            \item Nonfunctional Requirements
        \end{revisionitem}\\
        \midrule
        0.8 & C. W. Leung & 2025-02-10 & \begin{revisionitem}[Added:]
            \item Document Purpose
            \item Project Scope
            \item Intended Audience and Document Overview
            \item Document Conventions
            \item References and Acknowledgments
        \end{revisionitem}\\
        \midrule
        0.9 & H. T. Lei & 2025-02-10 & \begin{revisionitem}[Updated:]
            \item Refine Nonfunctional Requirements
        \end{revisionitem}\\
        \midrule
        1.0 & \begin{tabular}[t]{@{}l@{}}C. H. Yu\\ H. T. Lei\\ A. Enkhtaivan\\ H. K. Yum\\ C. W. Leung\end{tabular} & 2025-02-10 & \begin{revisionitem}[Updated:]
            \item Finalized draft with all requirements completed and reviewed
        \end{revisionitem}\\
        \midrule
        1.1 & C. W. Leung & 2025-04-28 & \begin{revisionitem}[Modified:]
            \item Updated requirements and features according to meeting1.docx instructions: roles, deleted/added features, error handling, credit system, page list, PICs.
        \end{revisionitem}\\
    \end{tabularx}
\end{center}

\clearpage
\pagenumbering{arabic}  % Switch to arabic numbers for main content

\chapter{Introduction}
\section{Document Purpose}
This document encompasses all essential functionalities,
including course delivery, user progress tracking, assessment tools, user and educator profiles, and credit/payment features. All requirements reflect the current system scope as revised per stakeholder and instructor feedback.

\section{Project Scope}
Kaiju Academy is an online self-learning platform that makes learning to code easier and more engaging. The platform delivers interactive coding courses for all skill levels through a self-paced learning environment with progress tracking and instant feedback. Features now focus on interactive code assessment, course credit/payment, and profile management. Community features, code competitions, dashboards, forum, notifications, calendar, keyboard shortcuts, and accessibility customizations have been removed from scope.

\section{Intended Audience and Document Overview}
This document serves as a guide for developers, managers, testers and stakeholders of Kaiju Academy. It contains sections covering product description, requirements, features, and non-functional requirements. We recommend starting with the product description section before moving to specific requirements.

\section{Definitions, Acronyms and Abbreviations}
\begin{center}
    \begin{tabularx}{\textwidth}{lX}
        \toprule
        \textbf{Abbreviation} & \textbf{Definition} \\
        \midrule
        2FA & Two-factor Authentication \\
        API & Application Programming Interface \\
        CDN & Content Delivery Networks \\
        GDPR & General Data Protection Regulation \\
        LMS & Learning Management System \\
        MC & Multiple Choice \\
        REST & Representational State Transfer \\
        SRS & Software Requirements Specification \\
        UI & User Interface \\
        UX & User Experience \\
        MFA & Multi-Factor Authentication \\
        \bottomrule
    \end{tabularx}
\end{center}

\section{Document Conventions}
This Project Requirements Specification adheres to the IEEE Std 830-1998 standard. The document utilizes Times New Roman font in size 11 for consistency and readability. Important terms are highlighted in bold, while supplementary notes are presented in italics. Each requirement is uniquely identified, with higher-level requirements inheriting their priority unless otherwise specified. Sections and requirements are sequentially numbered to facilitate easy navigation. Additionally, technical terms and acronyms are clearly defined in the Glossary to ensure clarity and understanding for all readers.

\section{References and Acknowledgments}
IEEE. IEEE Std 830-1998 IEEE Recommended Practice for Software Requirements Specifications. IEEE Computer Society, 1998.

\chapter{Overall Description}
\section{Product Overview}
Kaiju Academy is a web-based e-learning platform that provides interactive programming education through modern LMS capabilities and specialized coding features, facilitating the learning of programming anytime and anywhere as long as you are connected to the internet.

The platform serves as a comprehensive learning environment where students can engage with programming concepts through a structured curriculum. Users access the platform through web browsers, eliminating the need for software installation and reducing the barrier to entry for users.

The platform employs a module-based learning approach, where content is organized into discrete units that build upon each other. Each module typically contains:

\begin{itemize}
    \item Video lectures with synchronized transcripts
    \item Interactive coding exercises
    \item Supplementary reading materials
    \item Practice problems and assignments
    \item Progress assessments
\end{itemize}

Students can track their progress through a personalized user profile, which displays completed modules, registered courses, recommended courses, and achievement metrics. The integrated code assessment editor allows students to practice coding directly within the browser, with immediate feedback and automated grading capabilities.

Educators can utilize the platform to create courses, manage course materials, and monitor student progress. The system enables educators to update course modules, manage course content, and view taught courses with a sidebar of categories.

A new platform feature is the credit system, which allows users to purchase course access via payment and credits.

\section{Product Functionality}

Kaiju Academy provides the following major functionalities:
\begin{itemize}
    \item User Management and Authentication (including MFA if supported)
    \item Course Creation and Management
    \item Interactive Learning Environment (code assessment editor)
    \item Assessment and Progress Tracking
    \item User Profile (view all/registered/recommended courses, progress)
    \item Educator Profile (taught/manage courses, sidebar)
    \item Course Credit and Payment System
\end{itemize}

\section{User Classes and Characteristics}

Kaiju Academy serves three primary user classes:
\begin{description}
    \item[Admin:] Platform managers with full access for maintenance and monitoring.
    \item[User:] Students who access courses, register, track learning progress, and manage credits.
    \item[Teacher/Educator:] Subject matter experts who create/manage course content and monitor students.
\end{description}

\section{Design and Implementation Constraints}

The implementation of Kaiju Academy is governed by the following key constraints:

\begin{description}
    \item[Hardware Constraints]\mbox{}
        \begin{itemize}
            \item The platform should run with low memory usage.
        \end{itemize}
    \item[Performance Constraints]\mbox{}
        \begin{itemize}
            \item Page load time must not exceed 3 seconds.
            \item System must support at least 1000 concurrent users.
            \item Video streaming must adapt to user bandwidth.
            \item Code execution response time must be under 5 seconds.
            \item Database queries must complete within 1 second.
        \end{itemize}
    \item[Operational Constraints]\mbox{}
        \begin{itemize}
            \item System must achieve 99.9\% uptime.
            \item Must implement automated backup systems.
            \item Must support horizontal scaling.
            \item Must implement monitoring and logging.
            \item Must provide disaster recovery procedures.
        \end{itemize}
\end{description}

\section{Assumptions and Dependencies}

\subsection{Assumptions}
\begin{itemize}
    \item \textbf{Internet access:} Since the application relies on a web interface, users must have stable internet connectivity with minimum 5 Mbps bandwidth or access through institution's local network.
    \item \textbf{Minimum System Requirements:} All client devices must feature:
        \begin{itemize}
            \item Computer or mobile device with minimum 4GB RAM for smooth performance
            \item Web browser (Chrome, Firefox, Safari, Edge - latest 2 versions)
            \item HTML5 and JavaScript enabled
        \end{itemize}
\end{itemize}

\subsection{Dependencies}
\begin{itemize}
    \item \textbf{Third-Party Libraries and Services:}
        \begin{itemize}
            \item Authentication protocols (OAuth2)
            \item Database management systems (PostgreSQL, MongoDB, Redis)
            \item Cloud infrastructure services (AWS/Google Cloud/DigitalOcean)
            \item Content delivery networks (CDN)
            \item Video hosting platforms
            \item Code editor components (CodeMirror/Monaco)
        \end{itemize}
    \item \textbf{Development Methodologies:}
        \begin{itemize}
            \item UML Modeling: System architecture documentation and design specifications
            \item COMET Methodology: Concurrent object modeling for workflow integration
            \item Container orchestration with Docker and Kubernetes
            \item CI/CD pipeline implementation
        \end{itemize}
\end{itemize}

\subsection{Critical Risks}
\begin{itemize}
    \item \textbf{External System Failures:}
        \begin{itemize}
            \item Third-party service disruptions (authentication, video hosting)
            \item Cloud infrastructure outages
            \item CDN performance issues
            \item Database system failures
        \end{itemize}
\end{itemize}

\chapter{Specific Requirements}

\section{External Interface Requirements}
\subsection{User Interfaces}

Kaiju Academy provides a graphical user interface (GUI) tailored for users, educators, and admins. The UI components include:

\begin{description}
    \item[$\cdot$ Online Code Assessment Editor:] An interactive online editor for code assessment tasks, supporting in-browser code editing and running.
    \item[$\cdot$ Course Pages:] Structured pages for each course, displaying modules, materials (videos, PDFs), and assessments.
    \item[$\cdot$ User Profile:] View all available courses, registered courses (showing modules), recommended courses for registration, and progress tracking.
    \item[$\cdot$ Educator Profile:] View taught courses (with modules), manage courses with a sidebar of categories.
    \item[$\cdot$ Course Credit/Payment:] Users can purchase credits via payment to unlock course access.
    \item[$\cdot$ Login/Register:] Login (with MFA if available) and registration forms.
    \item[$\cdot$ Error Handling:] Consistent error messages for invalid inputs, failed logins, or system errors.
    \item[$\cdot$ Responsive Design:] All interfaces are accessible on desktops, tablets, and mobile devices.
\end{description}

\subsection{Hardware Interfaces}

The platform is designed to support various hardware components. Key considerations include:

\begin{description}
    \item[$\cdot$ User Devices:] Compatible with desktops, laptops, tablets, and smartphones supporting modern web browsers.
    \item[$\cdot$ Server Infrastructure:] Hosted on cloud-based servers (AWS, Google Cloud, or Azure) with auto-scaling.
    \item[$\cdot$ Peripheral Support:] Keyboards, mice, and touchscreens.
    \item[$\cdot$ Network Requirements:] Requires a stable internet connection for streaming and online coding.
\end{description}

\subsection{Software Interfaces}

The system will integrate with various software components to ensure smooth operation and functionality. These include:

\begin{description}
    \item[$\cdot$ Operating Systems:] Compatible with Windows, macOS, Linux, Android, and iOS.
    \item[$\cdot$ Web Browsers:] Supports the latest versions of Google Chrome, Mozilla Firefox, Safari, and Microsoft Edge.
    \item[$\cdot$ Database Management System:] Uses PostgreSQL or MySQL.
    \item[$\cdot$ Authentication Services:] Integration with OAuth2 and MFA where possible.
    \item[$\cdot$ APIs:] RESTful APIs for frontend/backend communication (user management, course, assessment, payment).
    \item[$\cdot$ Content Delivery Networks (CDN):] For efficient media delivery.
    \item[$\cdot$ Payment Gateway:] Integrated to support course credit purchases.
    \item[$\cdot$ Logging and Monitoring:] Centralized error and usage monitoring.
\end{description}

\section{Functional Requirements}

The following table lists all functional requirements kept in scope (removed: dashboard, forum, code competition, notification, calendar, keyboard shortcuts, data backup, screen reader, color blindness):

\begin{table}[h!]
    \centering
    \begin{tabular}{|c|l|p{10cm}|}
    \hline
    \textbf{Function ID} & \textbf{Function Name} & \textbf{Description} \\
    \hline
    FR01 & User Registration & Users can create an account by providing necessary details. \\
    FR02 & User Login & Users can log in using credentials; MFA supported if available. \\
    FR03 & Profile Management & Users can update personal details and account settings. \\
    FR04 & Role Management & Assign roles (User, Teacher, Admin) with specific permissions. \\
    FR05 & Course Creation & Educators can create and manage courses. \\
    FR06 & Course Enrollment & Users can register/enroll in courses (via credit/payment if required). \\
    FR07 & Course Content Management & Educators can upload and organize course materials. \\
    FR08 & Quiz/Code Assessment Management & Educators can create and assign coding/quiz assessments. \\
    FR09 & Automated Grading & System auto-grades multiple-choice, short-answer, and code assessment questions. \\
    FR10 & Manual Grading & Educators grade long-answer questions. \\
    FR11 & Progress Tracking & Users can track their learning progress in profile. \\
    FR12 & Course Credit/Payment & Users can buy credits to unlock courses. \\
    FR13 & Learning Recommendations & System recommends courses for registration. \\
    FR14 & Security & Implements authentication, authorization, and encryption. \\
    FR15 & Error Handling & Consistent user-facing error messages for invalid input, failed login, and system errors. \\
    \hline
    \end{tabular}
    \caption{Functional Requirements Table}
\end{table}

\section{Use Case Model}
\subsection{Use Case 1: Code Assessment}
\subsubsection{Description}
A user edits and runs code in an online editor as part of a course assessment.

\subsubsection{Actors and Preconditions}
\begin{itemize}
    \item \textbf{Actors}: User (Student)
    \item \textbf{Preconditions}: The user is logged in and enrolled in the course/module.
\end{itemize}

\subsubsection{Steps}
\begin{enumerate}
    \item The user opens the code assessment in a course/module.
    \item The user writes or edits code in the online editor.
    \item The user runs code to see results/output.
    \item The system provides immediate feedback or grading.
\end{enumerate}

\subsection{Use Case 2: Course Credit Purchase}
\subsubsection{Description}
A user buys credits to unlock/register for a course.

\subsubsection{Actors and Preconditions}
\begin{itemize}
    \item \textbf{Actors}: User (Student)
    \item \textbf{Preconditions}: The user is logged in; has access to payment methods.
\end{itemize}

\subsubsection{Steps}
\begin{enumerate}
    \item The user navigates to the credit purchase page.
    \item The user selects the amount of credits to buy.
    \item The user completes payment.
    \item Credits are added to the user’s account.
    \item The user uses credits to register for a course.
\end{enumerate}

\subsection{Use Case 3: Educator Course Management}
\subsubsection{Description}
An educator creates or manages courses and modules.

\subsubsection{Actors and Preconditions}
\begin{itemize}
    \item \textbf{Actors}: Teacher/Educator
    \item \textbf{Preconditions}: The educator is logged in and authorized.
\end{itemize}

\subsubsection{Steps}
\begin{enumerate}
    \item The educator opens the manage courses page (with sidebar categories).
    \item The educator creates/updates/deletes a course or module.
    \item The system validates data and updates course content.
    \item Students are notified of new/updated course materials (if applicable).
\end{enumerate}

\chapter{System Features}

This section describes the functional requirements of the product by its major features (all removed features are omitted):

\section{Online Code Assessment Editor}

\subsection{Description and Priority}
The code assessment editor is a high-priority feature, providing an online text editor where users can write, edit, and run code for assignments and practice within the course module.

\subsection{Stimulus/Response Sequences}
\begin{itemize}
    \item \textbf{User Action}: Opens code assessment in a module.\\
          \textbf{System Response}: Loads the online editor with problem description.
    \item \textbf{User Action}: Edits code and runs it.\\
          \textbf{System Response}: Executes code and displays result/output.
    \item \textbf{User Action}: Submits code for grading.\\
          \textbf{System Response}: Grades submission and provides feedback.
\end{itemize}

\subsection{Functional Requirements}
\begin{itemize}
    \item The code editor must support editing and running code in the browser.
    \item Students must receive immediate feedback on code execution.
    \item Submissions must be automatically graded.
\end{itemize}

\section{Course Pages and Profile Features}

\subsection{Course Pages}
\begin{itemize}
    \item Display list of modules, materials, and assessments.
    \item Allow access only to registered users (or after credit purchase).
\end{itemize}

\subsection{User Profile}
\begin{itemize}
    \item View all courses and registered courses (show modules).
    \item View recommended courses for registration.
    \item Track progress across all courses.
\end{itemize}

\subsection{Educator Profile}
\begin{itemize}
    \item View taught courses and modules.
    \item Manage course content (sidebar with categories).
\end{itemize}

\section{Course Credit and Payment}

\begin{itemize}
    \item Users can purchase credits via integrated payment gateway.
    \item Credits are used to register/unlock courses.
    \item System tracks user balance and transaction history.
\end{itemize}

\section{Authentication}

\begin{itemize}
    \item Users can register and log in.
    \item MFA is supported if available.
\end{itemize}

\section{Error Handling}
\begin{itemize}
    \item Consistent error messages for all invalid actions and system failures.
    \item Error handling for payment, registration, login, and code execution.
    \item Error handling is under the responsibility of Ankr.
\end{itemize}

\section{Frontend}
\begin{itemize}
    \item All frontend UI/UX development is overseen by Hazel and Daniel.
    \item Ensure all features/pages are responsive and user-friendly.
\end{itemize}

\chapter{Other Nonfunctional Requirements}

\section{Performance Requirements}
\begin{itemize}
    \item The platform must load within \textbf{2 seconds} for 95\% of users under normal load conditions.
    \item Code execution results must be returned within \textbf{5 seconds} for 99\% of submissions.
    \item The platform must support up to \textbf{10,000 concurrent users}.
    \item The platform must have an uptime of \textbf{99.9\%}.
\end{itemize}

\section{Safety Requirements}
\begin{itemize}
    \item The platform must prevent users from executing malicious code (sandboxing).
    \item The platform must comply with \textbf{GDPR} and \textbf{COPPA}.
\end{itemize}

\section{Security Requirements}
\begin{itemize}
    \item Users must authenticate via \textbf{OAuth 2.0} or email/password with \textbf{MFA} support.
    \item Role-based access control (User, Teacher, Admin).
    \item All sensitive data must be encrypted using \textbf{AES-256}.
    \item All data in transit must be secured using \textbf{TLS 1.2} or higher.
    \item The platform must prevent brute-force, SQL injection, and XSS attacks.
\end{itemize}

\section{Software Quality Attributes}
\begin{itemize}
    \item \textbf{Usability:} The platform must have an intuitive UI with a high satisfaction rate.
    \item \textbf{Maintainability:} The codebase must follow modular design principles.
    \item \textbf{Portability:} The platform must be compatible with major browsers and OSs.
    \item \textbf{Performance:} The platform must support \textbf{10,000 concurrent users} with a response time of less than \textbf{2 seconds}.
\end{itemize}

\section{Additional Requirements}
\begin{description}
    \item[$\cdot$ Database Requirements:] The system should support relational databases with ACID compliance.
    \item[$\cdot$ Internationalization:] The platform should support multiple languages, including English, Spanish, and Chinese.
    \item[$\cdot$ Legal Compliance:] The platform must adhere to data protection regulations (e.g., GDPR, CCPA).
    \item[$\cdot$ Reuse Objectives:] Modular architecture should be adopted to allow for code reuse across projects.
    \item[$\cdot$ Scalability Requirements:] The system should be designed to handle high concurrent users efficiently.
    \item[$\cdot$ Customization and Extensibility:] The system should allow customization for branding and feature addition.
\end{description}

\end{document}