\def\myversion{0.5}
\def\projectname{Kaiju Academy}
\date{}

\documentclass[a4paper, 11pt]{scrreprt}
\usepackage{srs}  % Load SRS document style

% Document-specific settings
\hypersetup{
    pdftitle={Software Requirement Specification},
    pdfauthor={C. H. Yu},
    pdfsubject={Technical Documentation},
    pdfkeywords={LaTeX, SRS, Software Requirements Specification, Software}
}

\begin{document}

\pagenumbering{roman}  % Start with roman numerals for front matter

\begin{titlepage}
    \begin{flushright}
        \rule{\textwidth}{5pt}\vskip1cm
        \begin{bfseries}
            \Huge{SOFTWARE REQUIREMENTS\\ SPECIFICATION}\\
            \vspace{1.6cm}
            for\\
            \vspace{1.6cm}
            \projectname\\  % Use project name variable
            \vspace{1.6cm}
            \LARGE{Version \myversion}\\
            \vspace{1.6cm}
            Prepared by\\
            Group A2\\
            \Large{
                \begin{tabularx}{\textwidth}{l l >{\raggedleft\arraybackslash}X}
                    YU Ching Hei & 1155193237 & \href{mailto:chyu@link.cuhk.edu.hk}{chyu@link.cuhk.edu.hk}\\
                    Lei Hei Tung & 1155194969 & \href{mailto:1155194969@link.cuhk.edu.hk}{1155194969@link.cuhk.edu.hk}\\
                    <name> & <student id> & <email>\\
                    <name> & <student id> & <email>\\
                    <name> & <student id> & <email>\\
                \end{tabularx}
            }\\
            \vspace{1.6cm}
            The Chinese University of Hong Kong\\
            Department of Computer Science and Engineering\\
            CSCI3100: Software Engineering\\
            \vspace{1.6cm}
            \today\\
        \end{bfseries}
    \end{flushright}
\end{titlepage}

\setuptoc{toc}{totoc}  % Add ToC to itself using KOMA-Script
\tableofcontents

\addchap{Document Revision History}  % Unnumbered chapter that appears in ToC

\begin{center}
    \begin{tabularx}{\textwidth}{>{\raggedright\arraybackslash}p{2cm}>{\raggedright\arraybackslash}p{3cm}>{\raggedright\arraybackslash}p{3cm}>{\raggedright\arraybackslash}X}
        \toprule
        Version & Revised By & Revision Date & Comments\\
        \midrule
        0.1 & C. H. Yu & 2025-02-08 & \begin{revisionitem}[Updated:]
            \item Initial document structure
            \item Basic template setup
        \end{revisionitem}\\
        \midrule
        0.2 & C. H. Yu & 2025-02-08 & \begin{revisionitem}[Updated:]
            \item Formatting
        \end{revisionitem}\\
        \midrule
        0.3 & C. H. Yu & 2025-02-08 & \begin{revisionitem}[Updated:]
            \item Titlepage formatting
            \item Page numbering
            \item Chapters title formatting
            \item Chapter and sections arrangement
        \end{revisionitem}\\
        \midrule
        0.4 & H. T. Lei & 2025-02-09 & \begin{revisionitem}[Added:]
            \item Specific requirements
        \end{revisionitem}\\
        \midrule
        0.4.1 & C. H. Yu & 2025-02-10 & \begin{revisionitem}[Updated:]
            \item Specific requirements: fixed compilation error
            \item Titlepage formatting: fixed alignment issue
        \end{revisionitem}\\
        \midrule
        0.5 & C. H. Yu & 2025-02-10 & \begin{revisionitem}[Added:]
            \item Acronyms and Abbreviations table
            \item Product Overview
            \item Product Functionality
            \item User Classes and Characteristics
            \item Design and Implementation Constraints
            \item Assumptions and Dependencies
        \end{revisionitem}\\
        \bottomrule
    \end{tabularx}
\end{center}

\clearpage
\pagenumbering{arabic}  % Switch to arabic numbers for main content

\chapter{Introduction}
\section{Document Purpose}
$<$Identify the product whose software requirements are specified in this document, 
including the revision or release number. Describe the scope of the product that is 
covered by this SRS, particularly if this SRS describes only part of the system or a 
single subsystem.$>$
TODO: Write 1-2 paragraphs describing the purpose of this document as explained above

\section{Project Scope}
$<$Provide a short description of the software being specified and its purpose, including relevant benefits, objectives, and goals. 
TODO: 1-2 paragraphs describing the scope of the product. Make sure to describe the benefits associated with the product.
$>$

\section{Intended Audience and Document Overview}
$<$Describe the different types of reader that the document is intended for, such as developers, project managers, marketing staff, users, testers, and documentation writers (In your case it would probably be the "client" and the professor). Describe what the rest of this SRS contains and how it is organized. Suggest a sequence for reading the document, beginning with the overview sections and proceeding through the sections that are most pertinent to each reader type.$>$

\section{Definitions, Acronyms and Abbreviations}
$<$Define all the terms necessary to properly interpret the SRS, including acronyms and abbreviations. You may wish to build a separate glossary that spans multiple projects or the entire organization, and just include terms specific to a single project in each SRS.
TODO: Please provide a list of all abbreviations and acronyms used in this document sorted in alphabetical order
$>$
\begin{center}
    \begin{tabularx}{\textwidth}{lX}
        \toprule
        \textbf{Abbreviation} & \textbf{Definition} \\
        \midrule
        AI & Artificial Intelligence \\
        API & Application Programming Interface \\
        CCPA & California Consumer Privacy Act \\
        CD & Continuous Deployment \\
        CDN & Content Delivery Networks \\
        CI & Continuous Integration \\
        CSS & Cascading Style Sheets \\
        FAQ & Frequently Asked Questions \\
        GDPR & General Data Protection Regulation \\
        GUI & Graphical User Interface \\
        HTML & HyperText Markup Language \\
        HTTP & HyperText Transfer Protocol \\
        IDE & Integrated Development Environment \\
        JS & JavaScript \\
        LMS & Learning Management System \\
        MC & Multiple Choice \\
        OS & Operating System \\
        REQ & Requirement \\
        REST & Representational State Transfer \\
        SRS & Software Requirements Specification \\
        SSL & Secure Sockets Layer \\
        TBD & To Be Determined \\
        TLS & Transport Layer Security \\
        UI & User Interface \\
        URL & Uniform Resource Locator \\
        UX & User Experience \\
        2FA & Two-factor Authentication \\
        \bottomrule
    \end{tabularx}
\end{center}


\section{Document Conventions}
$<$In general this document follows the IEEE formatting requirements. Use Arial font size 11, or 12 throughout the document for text. Use italics for comments. Document text should be single spaced and maintain the 1" margins found in this template. For Section and Subsection titles please follow the template. 

TODO: Describe any standards or typographical conventions that were followed when writing this SRS, such as fonts or highlighting that have special significance. Sometimes, it is useful to divide this section to several sections, e.g., Formatting Conventions, Naming Conventions, etc
$>$

\section{References and Acknowledgments}
$<$List any other documents or Web addresses to which this SRS refers. These may 
include user interface style guides, contracts, standards, system requirements 
specifications, use case documents, or a vision and scope document. Provide 
enough information so that the reader could access a copy of each reference, 
including title, author, version number, date, and source or location.$>$


\chapter{Overall Description}
\section{Product Overview}
% $<$Describe the context and origin of the product being specified in this SRS.  
% For example, state whether this product is a follow-on member of a product 
% family, a replacement for certain existing systems, or a new, self-contained 
% product. If the SRS defines a component of a larger system, relate the 
% requirements of the larger system to the functionality of this software and 
% identify interfaces between the two. A simple diagram that shows the major 
% components of the overall system, subsystem interconnections, and external 
% interfaces can be helpful.$>$

Kaiju Academy is a web-based e-learning platform that provides interactive programming education through modern LMS capabilities and specialized coding features, facilitating the learning of programming anytime and anywhere as long as you are connected to the internet.\\

The platform serves as a comprehensive learning environment where students can engage with programming concepts through a structured curriculum. Users access the platform through web browsers, eliminating the need for software installation reducing the barrier to entry for users. \\

The platform employs a module-based learning approach, where content is organized into discrete units that build upon each other. Each module typically contains:

\begin{itemize}
    \item Video lectures with synchronized transcripts
    \item Interactive coding exercises
    \item Supplementary reading materials
    \item Practice problems and assignments
    \item Progress assessments
\end{itemize}

Students can track their progress through a personalized dashboard, which displays completed modules, upcoming assignments, and achievement metrics. The integrated code editor allows students to practice coding directly within the browser, with immediate feedback and automated grading capabilities. \\

Educators can utilize the platform to create courses, manage course materials and provide targeted assistance where needed. They could also answer questions from students in the discussion forum. The system's analytics tools help identify areas where students may be struggling, enabling educators to provide additional teaching resources. The platform features automated grading and feedback for assignments, allowing educators to have more flexible arrangement. 

\section{Product Functionality}
% $<$Summarize the major functions the product must perform or must let the user 
% perform. Details will be provided in Section 3, so only a high level summary 
% (such as a bullet list) is needed here. Organize the functions to make them 
% understandable to any reader of the SRS. A picture of the major groups of 
% related requirements and how they relate, such as a top level data flow diagram 
% or object class diagram, is often effective.$>$

Kaiju Academy provides the following major functionalities:

\begin{itemize}
    \item User Management: Account creation, authentication, role-based access, and profile settings
    
    \item Course Management: Creation and organization of courses, content management, and enrollment tracking
    
    \item Learning Content: Video lectures, documents, quizzes, coding assignments, and progress tracking
    
    \item Coding Environment: Browser-based code editor with testing and feedback
    
    \item Discussion Features: Course forums, announcements, and community interaction
    
    \item Assessment Tools: Automated grading, progress tracking, and achievement system
    
    \item Administration: User management, content moderation, and system monitoring
\end{itemize}

\section{User Classes and Characteristics}
% $<$Identify the various user classes that you anticipate will use this product.  
% User classes may be differentiated based on frequency of use, subset of product 
% functions used, technical expertise, security or privilege levels, educational 
% level, or experience. Describe the pertinent characteristics of each user class.  
% Certain requirements may pertain only to certain user classes. Distinguish the 
% most important user classes for this product from those who are less important 
% to satisfy.$>$

Kaiju Academy serves several distinct user classes, each with specific characteristics and needs:

\begin{description}
    \item[Administrators:] Manage platform operations and user access
        \begin{itemize}[leftmargin=*]
            \item High technical expertise with system administration
            \item High privilage access for maintenance and monitoring
            \item Need complete system control and monitoring tools
            \item Require access to all platform features and settings
        \end{itemize}

    \item[Students:] Primary users of the learning platform
        \begin{itemize}[leftmargin=*]
            \item From various technical expertise levels (beginner to advanced)
            \item Regular access for course participation and completion
            \item Need intuitive interface and clear learning paths
            \item Need interactive learning environment
            \item Require progress tracking and performance feedback
        \end{itemize}
    
    \item[Educators:] Create and manage course content
        \begin{itemize}[leftmargin=*]
            \item Technical expertise in their teaching domains
            \item Regular access for content updates and student monitoring
            \item Need comprehensive content management tools
            \item Require analytics and student progress tracking capabilities
        \end{itemize}
    
    \item[Content Moderators:] Monitor and manage forum discussions
        \begin{itemize}[leftmargin=*]
            \item Moderate technical expertise required
            \item Regular access for content moderation
            \item Need moderation tools and user management features
            \item Require access to communication and reporting tools
        \end{itemize}
\end{description}

\section{Design and Implementation Constraints}
% $<$Describe any items or issues that will limit the options available to the 
% developers. These might include: corporate or regulatory policies; hardware 
% limitations (timing requirements, memory requirements); interfaces to other 
% applications; specific technologies, tools, and databases to be used; parallel 
% operations; language requirements; communications protocols; security 
% considerations; design conventions or programming standards (for example, if the 
% customer's organization will be responsible for maintaining the delivered 
% software).$>$

The implementation and ongoing development of Kaiju Academy shall be governed by the following technical and operational constraints:

\begin{description}
    \item[Hardware Constraints]\mbox{}
        \begin{itemize}
            \item TODO: Add hardware constraints
        \end{itemize}
    
    \item[Security Constraints]\mbox{}
        \begin{itemize}
            \item Must implement OAuth2 for authentication
            \item All communications must be encrypted using SSL/TLS
            \item User passwords must be hashed using industry-standard algorithms
            \item Code execution must be isolated in secure containers
            \item Regular security audits must be conducted
            \item Must comply with data protection regulations
        \end{itemize}
    
    \item[Performance Constraints]\mbox{}
        \begin{itemize}
            \item Page load time must not exceed 3 seconds
            \item System must support at least 1000 concurrent users
            \item Video streaming must adapt to user bandwidth
            \item Code execution response time must be under 5 seconds
            \item Database queries must complete within 1 second
        \end{itemize}
    
    \item[Development Constraints]\mbox{}
        \begin{itemize}
            \item Must follow Git version control practices
            \item Code must pass automated testing before deployment
            \item Must implement CI/CD pipeline
            \item Must follow RESTful API design principles
            \item Must maintain comprehensive API documentation
        \end{itemize}
    
    \item[Operational Constraints]\mbox{}
        \begin{itemize}
            \item System must achieve 99.9\% uptime
            \item Must implement automated backup systems
            \item Must support horizontal scaling
            \item Must implement monitoring and logging
            \item Must provide disaster recovery procedures
        \end{itemize}
\end{description}

\section{Assumptions and Dependencies}
% $<$List any assumed factors (as opposed to known facts) that could affect the 
% requirements stated in the SRS. These could include third-party or commercial 
% components that you plan to use, issues around the development or operating 
% environment, or constraints. The project could be affected if these assumptions 
% are incorrect, are not shared, or change. Also identify any dependencies the 
% project has on external factors, such as software components that you intend to 
% reuse from another project, unless they are already documented elsewhere (for 
% example, in the vision and scope document or the project plan).$>$

\subsection{Assumptions}
\begin{itemize}
    \item \textbf{Internet access:} Since the application relies on a web interface, users must have stable internet connectivity with minimum 5 Mbps bandwidth or access through institution's local network
    
    \item \textbf{Minimum System Requirements:} All client devices must feature:
        \begin{itemize}
            \item Computer or mobile device with minimum 4GB RAM for smooth performance
            \item Web browser (Chrome, Firefox, Safari, Edge - latest 2 versions)
            \item HTML5 and JavaScript enabled
        \end{itemize}
\end{itemize}

\subsection{Dependencies}
\begin{itemize}
    \item \textbf{Third-Party Libraries and Services:}
        \begin{itemize}
            \item Authentication protocols (OAuth2)
            \item Database management systems (PostgreSQL, MongoDB, Redis)
            \item Cloud infrastructure services (AWS/Google Cloud/DigitalOcean)
            \item Content delivery networks (CDN)
            \item Video hosting platforms
            \item Code editor components (CodeMirror/Monaco)
        \end{itemize}
    
    \item \textbf{Development Methodologies:}
        \begin{itemize}
            \item UML Modeling: System architecture documentation and design specifications
            \item COMET Methodology: Concurrent object modeling for workflow integration
            \item Container orchestration with Docker and Kubernetes
            \item CI/CD pipeline implementation
        \end{itemize}
\end{itemize}

\subsection{Critical Risks}
\begin{itemize}
    \item \textbf{External System Failures:}
        \begin{itemize}
            \item Third-party service disruptions (authentication, video hosting)
            \item Cloud infrastructure outages
            \item CDN performance issues
            \item Database system failures
        \end{itemize}
\end{itemize}


\chapter{Specific Requirements}

\section{External Interface Requirements}
\subsection{User Interfaces}
$<$Describe the logical characteristics of each interface between the software 
product and the users. This may include sample screen images, any GUI standards 
or product family style guides that are to be followed, screen layout 
constraints, standard buttons and functions (e.g., help) that will appear on 
every screen, keyboard shortcuts, error message display standards, and so on.  
Define the software components for which a user interface is needed. Details of 
the user interface design should be documented in a separate user interface 
specification.$>$

Kaiju Academy will have a graphical user interface (GUI) designed for ease of use and accessibility across different roles, including students, educators, and administrators. The UI components include:
\begin{description}
    \item[$\cdot$ Dashboard:] Upon login, users will be presented with a dashboard displaying their enrolled courses, upcoming deadlines, progress tracking, and notifications.
    \item[$\cdot$ Navigation:] A global navigation bar providing quick access to Courses, Assessments, Calendar, Discussion Forum, and Profile settings.
    \item[$\cdot$ Course Pages:] Each course contains a structured layout displaying modules, videos, PDFs, quizzes, and assessments with a progress tracker.
    \item[$\cdot$ Assessment Interface:] Interactive assessment screens allowing multiple-choice (MC) questions with self-checking, short-answer autograded questions, and long-answer questions submitted for manual grading.
    \item[$\cdot$ Progress Tracking:] A visual roadmap for students to track completed and pending modules and overall course progress.
    \item[$\cdot$ Discussion Forum:] A interactive discussion forum with search, filter, and sort options.
    \item[$\cdot$ Calendar:] An integrated calendar highlighting course schedules, assignment deadlines, and upcoming events.
    \item[$\cdot$ Notifications:] A notification center alerting users about new content, deadlines, and updates.
    \item[$\cdot$ Coding Environment] An embedded coding environment for practice exercises and coding competitions, similar to LeetCode.
    \item[$\cdot$ Error Handling] Consistent error messages displayed in case of invalid input, failed login attempts, or system errors.
    \item[$\cdot$ Keyboard Shortcuts] Common keyboard shortcuts for navigation and execution of commands, enhancing efficiency.
    \item[$\cdot$ Mobile Compatibility:] A responsive design ensuring accessibility on desktops, tablets, and mobile devices.
\end{description}

\subsection{Hardware Interfaces}
$<$Describe the logical and physical characteristics of each interface between 
the software product and the hardware components of the system. This may include 
the supported device types, the nature of the data and control interactions 
between the software and the hardware, and communication protocols to be 
used.$>$

The platform is designed to support various hardware components. Key considerations include:
\begin{description}
    \item[$\cdot$ User Devices:] The platform will be compatible with desktops, laptops, tablets, and smartphones supporting modern web browsers.
    \item[$\cdot$ Server Infrastructure:] Hosted on cloud-based servers (AWS, Google Cloud, or Azure) with auto-scaling to accommodate user load.
    \item[$\cdot$ Peripheral Support:] Users can interact using keyboards, mice, touchscreens, and audio devices for accessibility.
    \item[$\cdot$ Network Requirements:] Requires a stable internet connection with minimum bandwidth to support video streaming and coding environment interaction.
\end{description}

\subsection{Software Interfaces}
$<$Describe the connections between this product and other specific software 
components (name and version), including databases, operating systems, tools, 
libraries, and integrated commercial components. Identify the data items or 
messages coming into the system and going out and describe the purpose of each.  
Describe the services needed and the nature of communications. Refer to 
documents that describe detailed application programming interface protocols.  
Identify data that will be shared across software components. If the data 
sharing mechanism must be implemented in a specific way (for example, use of a 
global data area in a multitasking operating system), specify this as an 
implementation constraint.$>$

The system will integrate with various software components to ensure smooth operation and functionality. These include:
\begin{description}
    \item[$\cdot$ Operating Systems:] Compatible with Windows, macOS, Linux, Android, and iOS.
    \item[$\cdot$ Web Browsers:] Supports the latest versions of Google Chrome, Mozilla Firefox, Safari, and Microsoft Edge.
    \item[$\cdot$ Database Management System:] Uses PostgreSQL or MySQL to store user data, course materials, progress tracking, and assessment records.
    \item[$\cdot$ Authentication Services:] Integration with OAuth2 for third-party authentication (Google, GitHub, etc.) and two-factor authentication (2FA).
    \item[$\cdot$ APIs:] RESTful APIs to connect frontend and backend services, including: User authentication and management, Course content retrieval and management, Assessment grading and tracking, Notification and messaging services, Code execution API for online coding exercises
    \item[$\cdot$ Content Delivery Networks (CDN):] Utilized for efficient media delivery and reduced latency.
    \item[$\cdot$ Notification Services:] Integration with email and push notification services for alerts and reminders.
    \item[$\cdot$ Logging and Monitoring:] Implementation of centralized logging and monitoring tools for system health tracking.
\end{description}


\section{Functional Requirements}
$<$Functional requirements capture the intended behavior of the system. This 
behavior may be expressed as services, tasks or functions the system is required 
to perform. This section is the direct continuation of section 2.2 where you have 
specified the general functional requirements. Here, you should list in detail the
different product functions. $>$

Functional requirements capture the intended behavior of the system. This behavior may be expressed as services, tasks, or functions the system is required to perform. Below is a detailed list of product functions:

\begin{description}
    \item[$\cdot$ User Management:]
        \begin{enumerate}
            \item Users can sign up, log in, and update their profiles.
            \item Role-based access control (Student, Educator, Admin).
            \item Password recovery and security settings.
        \end{enumerate}
    \item[$\cdot$ Course Management:]
        \begin{enumerate}
            \item Educators can create, update, and delete courses.
            \item Upload and manage course materials (videos, PDFs, quizzes).
            \item Students can enroll, drop, and track their progress.
        \end{enumerate}
    \item[$\cdot$ Quiz and Assessment:]
        \begin{enumerate}
            \item MC questions with self-checking and automatic grading.
            \item Short-answer questions (some autograded, some educator-reviewed).
            \item Long-answer questions assigned to educators for grading.
            \item Popup MC questions during the course for engagement.
        \end{enumerate}
    \item[$\cdot$ Progress Tracking:]
        \begin{enumerate}
            \item Roadmap displaying completed and pending modules.
            \item Percentage-based completion tracker.
            \item Assessment performance tracking.
        \end{enumerate}
    \item[$\cdot$ Notifications:]
        \begin{enumerate} 
            \item Automated alerts for deadlines, new content, and assessments.
            \item Customizable notification preferences.
        \end{enumerate}
    \item[$\cdot$ Search and Filter:] Users can search and filter courses, discussions, and assessments.
    \item[$\cdot$ Calendar Integration:] Displays course schedules, assignment deadlines, and learning reminders.
    \item[$\cdot$ Discussion Forum:] Users can ask questions, respond, and interact with educators.
    \item[$\cdot$ Daily Assessment \& Learning Reminders:] Personalized daily learning tasks and notifications.
    \item[$\cdot$ Learning Path Review \& Recommendations:] System-generated personalized course suggestions based on progress.
    \item[$\cdot$ Online Coding Judge:] Users can solve coding problems with real-time execution and have coding competitions with leaderboard tracking.
\end{description}

\section{Use Case Model}
$<$This section is the direct continuation of section 2.3 where you have specified 
the general use cases. Here, you should list in detail the different use cases. 
$>$
\subsection{Use Case \#1}
$<$Describe the use case in detail.$>$

\subsection{Use Case \#2}
$<$Describe the use case in detail.$>$


\chapter{System Features}
$<$This template illustrates organizing the functional requirements for the 
product by system features, the major services provided by the product. You may 
prefer to organize this section by use case, mode of operation, user class, 
object class, functional hierarchy, or combinations of these, whatever makes the 
most logical sense for your product.$>$

\section{System Feature 1}
$<$Don't really say "System Feature 1." State the feature name in just a few 
words.$>$

\subsection{Description and Priority}
$<$Provide a short description of the feature and indicate whether it is of 
High, Medium, or Low priority. You could also include specific priority 
component ratings, such as benefit, penalty, cost, and risk (each rated on a 
relative scale from a low of 1 to a high of 9).$>$

\subsection{Stimulus/Response Sequences}
$<$List the sequences of user actions and system responses that stimulate the 
behavior defined for this feature. These will correspond to the dialog elements 
associated with use cases.$>$

\subsection{Functional Requirements}
$<$Itemize the detailed functional requirements associated with this feature.  
These are the software capabilities that must be present in order for the user 
to carry out the services provided by the feature, or to execute the use case.  
Include how the product should respond to anticipated error conditions or 
invalid inputs. Requirements should be concise, complete, unambiguous, 
verifiable, and necessary. Use "TBD" as a placeholder to indicate when necessary 
information is not yet available.$>$

$<$Each requirement should be uniquely identified with a sequence number or a 
meaningful tag of some kind.$>$

REQ-1:	REQ-2:

\section{System Feature 2 (and so on)}


\chapter{Other Nonfunctional Requirements}

\section{Performance Requirements}
$<$If there are performance requirements for the product under various 
circumstances, state them here and explain their rationale, to help the 
developers understand the intent and make suitable design choices. Specify the 
timing relationships for real time systems. Make such requirements as specific 
as possible. You may need to state performance requirements for individual 
functional requirements or features.$>$

\section{Safety Requirements}
$<$Specify those requirements that are concerned with possible loss, damage, or 
harm that could result from the use of the product. Define any safeguards or 
actions that must be taken, as well as actions that must be prevented. Refer to 
any external policies or regulations that state safety issues that affect the 
product's design or use. Define any safety certifications that must be 
satisfied.$>$

\section{Security Requirements}
$<$Specify any requirements regarding security or privacy issues surrounding use 
of the product or protection of the data used or created by the product. Define 
any user identity authentication requirements. Refer to any external policies or 
regulations containing security issues that affect the product. Define any 
security or privacy certifications that must be satisfied.$>$

\section{Software Quality Attributes}
$<$Specify any additional quality characteristics for the product that will be 
important to either the customers or the developers. Some to consider are: 
adaptability, availability, correctness, flexibility, interoperability, 
maintainability, portability, reliability, reusability, robustness, testability, 
and usability. Write these to be specific, quantitative, and verifiable when 
possible. At the least, clarify the relative preferences for various attributes, 
such as ease of use over ease of learning.$>$

\section{Business Rules}
$<$List any operating principles about the product, such as which individuals or 
roles can perform which functions under specific circumstances. These are not 
functional requirements in themselves, but they may imply certain functional 
requirements to enforce the rules.$>$


\chapter{Other Requirements}
$<$Define any other requirements not covered elsewhere in the SRS. This might 
include database requirements, internationalization requirements, legal 
requirements, reuse objectives for the project, and so on. Add any new sections 
that are pertinent to the project.$>$

\addchap{Appendix A: Glossary}
%see https://en.wikibooks.org/wiki/LaTeX/Glossary
$<$Define all the terms necessary to properly interpret the SRS, including 
acronyms and abbreviations. You may wish to build a separate glossary that spans 
multiple projects or the entire organization, and just include terms specific to 
a single project in each SRS.$>$

\addchap{Appendix B: Analysis Models}
$<$Optionally, include any pertinent analysis models, such as data flow 
diagrams, class diagrams, state-transition diagrams, or entity-relationship 
diagrams.$>$

\addchap{Appendix C: To Be Determined List}
$<$Collect a numbered list of the TBD (to be determined) references that remain 
in the SRS so they can be tracked to closure.$>$

\end{document}