\documentclass[a4paper, 11pt]{scrreprt}
\usepackage{listings}
\usepackage{underscore}
\usepackage{tabularx}  % Add tabularx package
\usepackage{enumitem}  % Add enumitem package for list spacing control
\usepackage{booktabs}  % Add booktabs package
\usepackage{mathptmx}  % Times New Roman font for text and math
\usepackage[T1]{fontenc}  % Better font encoding
\usepackage[bookmarks=true]{hyperref}
\usepackage[utf8]{inputenc}
\usepackage[english]{babel}
\usepackage{scrlayer-scrpage}
\hypersetup{
    pdftitle={Software Requirement Specification},    % title
    pdfauthor={C. H. Yu},                     % author
    pdfsubject={Technical Documentation},                        % subject of the document
    pdfkeywords={LaTeX, SRS, Software Requirements Specification, Software}, % list of keywords
    colorlinks=true,        % false: boxed links; true: colored links
    linkcolor=black,         % color of internal links
    citecolor=black,        % color of links to bibliography
    filecolor=black,        % color of file links
    urlcolor=black,        % color of external links
    linktoc=all             % both page numbers and text in ToC are linked
}%
\def\myversion{0.2}
\def\projectname{Kaiju Academy}  % Define project name variable
\date{}
%\title
\usepackage{hyperref}

% Setup headers and footers
\automark[chapter]{chapter}  % Automatically mark chapters for header

% Configure page layout
\KOMAoptions{
    headheight=1.5cm,      % Height of the header
    footheight=1.2cm,      % Height of the footer
    % headsepline=true,      % Enable header separator line
    % footsepline=true,      % Enable footer separator line
    % headwidth=\linewidth,  % Width of header
    % footwidth=\linewidth   % Width of footer
}

% Set separator line thickness

% Set margins
\usepackage[left=2cm,right=2cm,top=2.5cm,bottom=3cm]{geometry}

%%%%%%%%%%%%%%%%%%%%%%%%%%%%%%%%%%%%%%%%
% Header and footer content
\ihead{Software Requirements Specification}  % Inner header (left on odd, right on even pages)
\chead{}  % Center header (empty)
\ohead{\headmark}  % Outer header (right on odd, left on even pages)
% \lehead{}  % Left header on even pages
% \cehead{}  % Center header on even pages (empty)
% \rehead{}  % Right header on even pages
% \lohead{}  % Left header on odd pages
% \cohead{}  % Center header on odd pages (empty)
% \rohead{}  % Right header on odd pages
\ifoot{}  % Inner footer (empty)
\cfoot{}  % Center footer (empty)
\ofoot{Page \pagemark}  % Outer footer with page number
\setkomafont{pagehead}{\normalfont}  % Set header font
\setkomafont{headsepline}{\color{black}\rule{\textwidth}{1pt}}  % Header line thickness
\setkomafont{footsepline}{\color{black}\rule{\textwidth}{1pt}}  % Footer line thickness
%%%%%%%%%%%%%%%%%%%%%%%%%%%%%%%%%%%%%%%%

% Ensure consistent page style on chapter pages
\renewcommand*{\chapterpagestyle}{scrheadings}

%%%%%%%%%%%%%%%%%%%%%%%%%%%%%%%%%%%%%%%%
% Set chapter spacing
\RedeclareSectionCommand[
  beforeskip=0pt,     % Space before chapter title
  afterskip=1cm,      % Space after chapter title
  innerskip=1cm,      % Space between chapter number and title
  font=\normalfont    % Use Times New Roman (inherited from mathptmx)
]{chapter}

%%%%%%%%%%%%%%%%%%%%%%%%%%%%%%%%%%%%%%%%
% Set section title fonts
\setkomafont{chapter}{\normalfont\huge\bfseries}  % Chapter titles
\setkomafont{section}{\normalfont\Large\bfseries}  % Section titles
\setkomafont{subsection}{\normalfont\large\bfseries}  % Subsection titles
\setkomafont{subsubsection}{\normalfont\normalsize\bfseries}  % Subsubsection titles
\setkomafont{paragraph}{\normalfont\normalsize\bfseries}  % Paragraph titles
\setkomafont{subparagraph}{\normalfont\normalsize\bfseries}  % Subparagraph titles

%%%%%%%%%%%%%%%%%%%%%%%%%%%%%%%%%%%%%%%%
% Table of contents formatting
\addtokomafont{disposition}{\rmfamily}  % Set Times New Roman for ToC
\setkomafont{descriptionlabel}{\normalfont\bfseries}  % Description labels
\setkomafont{dictum}{\normalfont\small}  % Dictum text
\setkomafont{captionlabel}{\normalfont\bfseries}  % Caption labels
\setkomafont{caption}{\normalfont\small}  % Caption text

%%%%%%%%%%%%%%%%%%%%%%%%%%%%%%%%%%%%%%%%

%%%%%%%%%%%%%%%%%%%%%%%%%%%%%%%%%%%%%%%%
% Define revision history item environment
\newenvironment{revisionitem}[1][]{%
    \begin{minipage}[t]{\linewidth}%
        #1
        \begin{itemize}[
            itemsep=0pt,
            parsep=0pt,
            topsep=0pt,
            leftmargin=*,
            labelsep=0em,
            label=\textendash
        ]%
}{%
        \end{itemize}%
    \end{minipage}%
}
%%%%%%%%%%%%%%%%%%%%%%%%%%%%%%%%%%%%%%%%

\begin{document}

\pagenumbering{roman}  % Start with roman numerals for front matter

\begin{titlepage}
    \begin{flushright}
        \rule{\textwidth}{5pt}\vskip1cm
        \begin{bfseries}
            \Huge{SOFTWARE REQUIREMENTS\\ SPECIFICATION}\\
            \vspace{1.6cm}
            for\\
            \vspace{1.6cm}
            \projectname\\  % Use project name variable
            \vspace{1.6cm}
            \LARGE{Version \myversion}\\
            \vspace{1.6cm}
            Prepared by\\
            Group A2\\
            \begin{tabularx}{\textwidth}{X l X}
            YU Ching Hei & 1155193237 & \href{mailto:chyu@link.cuhk.edu.hk}{chyu@link.cuhk.edu.hk}\\
            Leung Chung Wang & 1155194650 & \href{mailto:1155194650@link.cuhk.edu.hk}{1155194650@link.cuhk.edu.hk}\\
            <name> & <student id> & <email>\\
            <name> & <student id> & <email>\\
            \end{tabularx}\\
            \vspace{1.6cm}
            The Chinese University of Hong Kong\\
            Department of Computer Science and Engineering\\
            CSCI3100: Software Engineering\\
            \vspace{1.6cm}
            \today\\
        \end{bfseries}
    \end{flushright}
\end{titlepage}

\tableofcontents


\chapter{Document Revision History}

\begin{center}
    \begin{tabularx}{\textwidth}{>{\raggedright\arraybackslash}p{2cm}>{\raggedright\arraybackslash}p{3cm}>{\raggedright\arraybackslash}p{3cm}>{\raggedright\arraybackslash}X}
        \toprule
        Version & Revised By & Revision Date & Comments\\
        \midrule
        0.1 & C. H. Yu & 2025-02-08 & \begin{revisionitem}[Updated:]
            \item Initial document structure
            \item Basic template setup
        \end{revisionitem}\\
        0.2 & C. H. Yu & 2025-02-08 & \begin{revisionitem}[Updated:]
            \item Formatting
        \end{revisionitem}\\
        \bottomrule
    \end{tabularx}
\end{center}

\clearpage
\pagenumbering{arabic}  % Switch to arabic numbers for main content

\chapter{Introduction}

\section{Purpose}
This document outlines the software requirements for Kaiju Academy, an online self-learning platform for coding education, Version 0.2 and prepared by Group A2. The document encompasses all essential functionalities, including course delivery, user progress tracking, community interaction, and assessment tools, providing both high-level and specific requirements.  This document shall form the basis for all stakeholders and developers to understand intended features and constraints of the platform.

\section{Document Conventions}
This Project Requirements Specification adheres to the IEEE Std 830-1998 standard.It utilizes Times New Roman font in size 11 for consistency and readability. Important terms are highlighted in bold, while supplementary notes are presented in italics. Each requirement is uniquely identified (e.g., R1), with higher-level requirements inheriting their priority unless otherwise specified. Sections and requirements are sequentially numbered (e.g., 5.1.1) to facilitate easy navigation. Additionally, technical terms and acronyms are clearly defined in the Glossary to ensure clarity and understanding for all readers.
\section{Intended Audience and Reading Suggestions}
This document is intended for a diverse audience, including software developers, project managers, quality assurance testers, and stakeholders involved in the Kaiju Academy project, organized into sections that detail the overall description of the product, specific requirements, system features, and non-functional requirements. Readers are encouraged to start with the overall description to understand the context and then proceed to the specific requirements relevant to their roles.

\section{Project Scope}
Kaiju Academy is an innovative online self-learning platform aiming to innovate code education. The aspire of Kaiju is to make users' learning of programming easier and more interactive in course-and-assessment-driven environment with community support, enhancing accessibility as users can learn at their own pace, while interactive and game-like experiences drive engagement and completion. Key objectives include a wide variety of coding courses, progress tracking, and instant feedback through assessments.\\\\
Kaiju Academy aligns with corporate goals of expanding digital education and enhancing user engagement through technology-driven solutions, supporting the development of a skilled workforce ready for the digital age.


\section{References}
IEEE. IEEE Std 830-1998 IEEE Recommended Practice for Software Requirements Specifications. IEEE Computer Society, 1998.


\chapter{Overall Description}

\section{Product Perspective}
$<$Describe the context and origin of the product being specified in this SRS.  
For example, state whether this product is a follow-on member of a product 
family, a replacement for certain existing systems, or a new, self-contained 
product. If the SRS defines a component of a larger system, relate the 
requirements of the larger system to the functionality of this software and 
identify interfaces between the two. A simple diagram that shows the major 
components of the overall system, subsystem interconnections, and external 
interfaces can be helpful.$>$

\section{Product Functions}
$<$Summarize the major functions the product must perform or must let the user 
perform. Details will be provided in Section 3, so only a high level summary 
(such as a bullet list) is needed here. Organize the functions to make them 
understandable to any reader of the SRS. A picture of the major groups of 
related requirements and how they relate, such as a top level data flow diagram 
or object class diagram, is often effective.$>$

\section{User Classes and Characteristics}
$<$Identify the various user classes that you anticipate will use this product.  
User classes may be differentiated based on frequency of use, subset of product 
functions used, technical expertise, security or privilege levels, educational 
level, or experience. Describe the pertinent characteristics of each user class.  
Certain requirements may pertain only to certain user classes. Distinguish the 
most important user classes for this product from those who are less important 
to satisfy.$>$

\section{Operating Environment}
$<$Describe the environment in which the software will operate, including the 
hardware platform, operating system and versions, and any other software 
components or applications with which it must peacefully coexist.$>$

\section{Design and Implementation Constraints}
$<$Describe any items or issues that will limit the options available to the 
developers. These might include: corporate or regulatory policies; hardware 
limitations (timing requirements, memory requirements); interfaces to other 
applications; specific technologies, tools, and databases to be used; parallel 
operations; language requirements; communications protocols; security 
considerations; design conventions or programming standards (for example, if the 
customer's organization will be responsible for maintaining the delivered 
software).$>$

\section{User Documentation}
$<$List the user documentation components (such as user manuals, on-line help, 
and tutorials) that will be delivered along with the software. Identify any 
known user documentation delivery formats or standards.$>$
\section{Assumptions and Dependencies}

$<$List any assumed factors (as opposed to known facts) that could affect the 
requirements stated in the SRS. These could include third-party or commercial 
components that you plan to use, issues around the development or operating 
environment, or constraints. The project could be affected if these assumptions 
are incorrect, are not shared, or change. Also identify any dependencies the 
project has on external factors, such as software components that you intend to 
reuse from another project, unless they are already documented elsewhere (for 
example, in the vision and scope document or the project plan).$>$


\chapter{External Interface Requirements}

\section{User Interfaces}
$<$Describe the logical characteristics of each interface between the software 
product and the users. This may include sample screen images, any GUI standards 
or product family style guides that are to be followed, screen layout 
constraints, standard buttons and functions (e.g., help) that will appear on 
every screen, keyboard shortcuts, error message display standards, and so on.  
Define the software components for which a user interface is needed. Details of 
the user interface design should be documented in a separate user interface 
specification.$>$

\section{Hardware Interfaces}
$<$Describe the logical and physical characteristics of each interface between 
the software product and the hardware components of the system. This may include 
the supported device types, the nature of the data and control interactions 
between the software and the hardware, and communication protocols to be 
used.$>$

\section{Software Interfaces}
$<$Describe the connections between this product and other specific software 
components (name and version), including databases, operating systems, tools, 
libraries, and integrated commercial components. Identify the data items or 
messages coming into the system and going out and describe the purpose of each.  
Describe the services needed and the nature of communications. Refer to 
documents that describe detailed application programming interface protocols.  
Identify data that will be shared across software components. If the data 
sharing mechanism must be implemented in a specific way (for example, use of a 
global data area in a multitasking operating system), specify this as an 
implementation constraint.$>$

\section{Communications Interfaces}
$<$Describe the requirements associated with any communications functions 
required by this product, including e-mail, web browser, network server 
communications protocols, electronic forms, and so on. Define any pertinent 
message formatting. Identify any communication standards that will be used, such 
as FTP or HTTP. Specify any communication security or encryption issues, data 
transfer rates, and synchronization mechanisms.$>$


\chapter{System Features}
$<$This template illustrates organizing the functional requirements for the 
product by system features, the major services provided by the product. You may 
prefer to organize this section by use case, mode of operation, user class, 
object class, functional hierarchy, or combinations of these, whatever makes the 
most logical sense for your product.$>$

\section{System Feature 1}
$<$Don't really say "System Feature 1." State the feature name in just a few 
words.$>$

\subsection{Description and Priority}
$<$Provide a short description of the feature and indicate whether it is of 
High, Medium, or Low priority. You could also include specific priority 
component ratings, such as benefit, penalty, cost, and risk (each rated on a 
relative scale from a low of 1 to a high of 9).$>$

\subsection{Stimulus/Response Sequences}
$<$List the sequences of user actions and system responses that stimulate the 
behavior defined for this feature. These will correspond to the dialog elements 
associated with use cases.$>$

\subsection{Functional Requirements}
$<$Itemize the detailed functional requirements associated with this feature.  
These are the software capabilities that must be present in order for the user 
to carry out the services provided by the feature, or to execute the use case.  
Include how the product should respond to anticipated error conditions or 
invalid inputs. Requirements should be concise, complete, unambiguous, 
verifiable, and necessary. Use "TBD" as a placeholder to indicate when necessary 
information is not yet available.$>$

$<$Each requirement should be uniquely identified with a sequence number or a 
meaningful tag of some kind.$>$

REQ-1:	REQ-2:

\section{System Feature 2 (and so on)}


\chapter{Other Nonfunctional Requirements}

\section{Performance Requirements}
$<$If there are performance requirements for the product under various 
circumstances, state them here and explain their rationale, to help the 
developers understand the intent and make suitable design choices. Specify the 
timing relationships for real time systems. Make such requirements as specific 
as possible. You may need to state performance requirements for individual 
functional requirements or features.$>$

\section{Safety Requirements}
$<$Specify those requirements that are concerned with possible loss, damage, or 
harm that could result from the use of the product. Define any safeguards or 
actions that must be taken, as well as actions that must be prevented. Refer to 
any external policies or regulations that state safety issues that affect the 
product's design or use. Define any safety certifications that must be 
satisfied.$>$

\section{Security Requirements}
$<$Specify any requirements regarding security or privacy issues surrounding use 
of the product or protection of the data used or created by the product. Define 
any user identity authentication requirements. Refer to any external policies or 
regulations containing security issues that affect the product. Define any 
security or privacy certifications that must be satisfied.$>$

\section{Software Quality Attributes}
$<$Specify any additional quality characteristics for the product that will be 
important to either the customers or the developers. Some to consider are: 
adaptability, availability, correctness, flexibility, interoperability, 
maintainability, portability, reliability, reusability, robustness, testability, 
and usability. Write these to be specific, quantitative, and verifiable when 
possible. At the least, clarify the relative preferences for various attributes, 
such as ease of use over ease of learning.$>$

\section{Business Rules}
$<$List any operating principles about the product, such as which individuals or 
roles can perform which functions under specific circumstances. These are not 
functional requirements in themselves, but they may imply certain functional 
requirements to enforce the rules.$>$


\chapter{Other Requirements}
$<$Define any other requirements not covered elsewhere in the SRS. This might 
include database requirements, internationalization requirements, legal 
requirements, reuse objectives for the project, and so on. Add any new sections 
that are pertinent to the project.$>$

\section{Appendix A: Glossary}
%see https://en.wikibooks.org/wiki/LaTeX/Glossary
$<$Define all the terms necessary to properly interpret the SRS, including 
acronyms and abbreviations. You may wish to build a separate glossary that spans 
multiple projects or the entire organization, and just include terms specific to 
a single project in each SRS.$>$

\section{Appendix B: Analysis Models}
$<$Optionally, include any pertinent analysis models, such as data flow 
diagrams, class diagrams, state-transition diagrams, or entity-relationship 
diagrams.$>$

\section{Appendix C: To Be Determined List}
$<$Collect a numbered list of the TBD (to be determined) references that remain 
in the SRS so they can be tracked to closure.$>$

\end{document}