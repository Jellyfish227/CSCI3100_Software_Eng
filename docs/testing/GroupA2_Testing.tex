\def\myversion{0.1}
\def\projectname{Kaiju Academy}

\documentclass[a4paper,11pt]{scrartcl}
\usepackage{testing} 
\usepackage{enumitem}
\setlist[itemize]{noitemsep,topsep=0pt,parsep=0pt,partopsep=0pt}
\hypersetup{
    pdftitle={Test Plan for \projectname},
    pdfauthor={Group A2},
    pdfsubject={Test Plan Documentation},
    pdfkeywords={LaTeX, Test Plan, Software Engineering, Test, CSCI3100}
}

\begin{document}

\pagenumbering{roman}

% --- Title Page ---
\begin{titlepage}
    \begin{flushright}
        \rule{\textwidth}{5pt}\vskip1cm
        \begin{bfseries}
            \Huge{TEST PLAN}\\
            \vspace{1.6cm}
            for\\
            \vspace{1.6cm}
            \projectname\\
            \vspace{1.6cm}
            \LARGE{Version \myversion}\\
            \vspace{1.6cm}
            Prepared by\\
            Group A2\\
            \Large{
                \begin{tabularx}{\textwidth}{l l >{\raggedleft\arraybackslash}X}
                    YU Ching Hei & 1155193237 & \href{mailto:chyu@link.cuhk.edu.hk}{chyu@link.cuhk.edu.hk}\\
                    Lei Hei Tung & 1155194969 & \href{mailto:1155194969@link.cuhk.edu.hk}{1155194969@link.cuhk.edu.hk}\\
                    Ankhbayar Enkhtaivan & 1155185142 & \href{mailto:1155185142@link.cuhk.edu.hk}{1155185142@link.cuhk.edu.hk}\\
                    Yum Ho Kan & 1155195234 & \href{mailto:1155195234@link.cuhk.edu.hk}{1155195234@link.cuhk.edu.hk}\\
                    Leung Chung Wang & 1155194650 & \href{mailto:1155194650@link.cuhk.edu.hk}{1155194650@link.cuhk.edu.hk}\\
                \end{tabularx}
            }\\
            \vspace{1.6cm}
            The Chinese University of Hong Kong\\
            Department of Computer Science and Engineering\\
            CSCI3100: Software Engineering\\
            \vspace{1.6cm}
            \today\\
        \end{bfseries}
    \end{flushright}
\end{titlepage}

% --- Table of Contents ---
\section*{Contents}
\addcontentsline{toc}{section}{Contents}
\tableofcontents
\clearpage

% --- Document Revision History ---
\section*{Document Revision History}
\addcontentsline{toc}{section}{Document Revision History}
\begin{center}
\begin{tabularx}{\textwidth}{l l l X}
\toprule
Version & Revised By & Revision Date & Comments \\
\midrule
0.1 & C. W. Leung & 2025-05-06 & Initial draft for Kaiju Academy Test Plan \\
\bottomrule

\end{tabularx}
\end{center}

\clearpage
\pagenumbering{arabic}

%----------------- Main Content Starts Here -----------------%

\section{Introduction}

\subsection{Purpose}
This document defines the test plan for the Kaiju Academy online learning platform. It describes the scope, objectives, test cases, resources, approach, schedule, risks, and reporting methods for the project.

\subsection{References and Acknowledgments}
\begin{itemize}[leftmargin=*]
    \item Kaiju Academy Software Requirements Specification v1.1
    \item Kaiju Academy Design and Implementation v1.1
    \item This document was prepared with the assistance of AI tools (e.g., ChatGPT 4.1) for drafting and review.
\end{itemize}

\section{Scope and Objectives}

\subsection{Scope}
Testing will cover the following key areas of Kaiju Academy:
\begin{itemize}[leftmargin=*]
    \item User registration, login, authentication (including MFA)
    \item Course browsing, enrollment, and access
    \item Interactive code assessment and automated grading
    \item Profile and progress tracking (student and educator)
    \item Educator course and material management
    \item Error handling, security, and data integrity
    \item Responsive UI/UX across devices (desktop, tablet, mobile)
    \item Licence management: entering and validating a licence key before accessing course content.
\end{itemize}

\subsection{Out of Scope}
\begin{itemize}[leftmargin=*]
    \item Community features (forum, dashboard, notifications, calendar)
    \item Credit/Payment purchasing and related flows
    \item Modding/hacking scenarios and hardware-specific edge cases
    \item Third-party payment gateway/internal payment logic
\end{itemize}

\subsection{Objectives}
\begin{itemize}[leftmargin=*]
    \item Verify that all main user stories and requirements are implemented and work as intended
    \item Ensure the platform meets performance, usability, and security expectations
    \item Confirm the system is stable and ready for release
    \item Validate integration with external modules (SurrealDB, AWS)
\end{itemize}

\section{Test Cases and Scenarios}

\subsection{Functional Test Cases}
To ensure each feature works as expected based on system requirements. These tests validate that the platform behaves correctly under normal and edge-case scenarios.
\begin{enumerate}[leftmargin=*]
    \item \textbf{User Registration and Login}
        \begin{itemize}
            \item Steps: Register with email, verify email, login with password (with/without MFA)
            \item Expected Result: Account created, verification email sent, login succeeds or fails as appropriate
            \item Pass/Fail: Account appears in database, valid JWT issued, protected endpoints accessible after login
        \end{itemize}
    \item \textbf{Course Browsing and Enrollment}
        \begin{itemize}
            \item Steps: Browse course catalog, enroll in available courses
            \item Expected Result: Enrollment successful, course appears in user’s profile
            \item Pass/Fail: Enrolled courses listed, correct access to course content
        \end{itemize}
    \item \textbf{Interactive Code Assessment}
        \begin{itemize}
            \item Steps: Access code editor, submit code, receive auto-grading and result
            \item Expected Result: Code executes, output and feedback displayed, grade recorded
            \item Pass/Fail: Output matches expected, grade visible in profile
        \end{itemize}
    \item \textbf{Educator Course Management}
        \begin{itemize}
            \item Steps: Create/update/delete course, upload materials
            \item Expected Result: Changes reflected for students, materials accessible
            \item Pass/Fail: CRUD operations function, permissions enforced
        \end{itemize}
    \item \textbf{Profile and Progress Tracking}
        \begin{itemize}
            \item Steps: Complete modules, view progress page
            \item Expected Result: Progress updates in real time
            \item Pass/Fail: Progress bar/percentage accurate, completed modules listed
        \end{itemize}
        \item \textbf{Licence Management}
        \begin{itemize}
            \item Steps: Attempt to access system features without a licence key; enter invalid key; enter valid key.
            \item Expected Result: Access denied until a valid key is provided; valid key grants access.
            \item Pass/Fail: System restricts access appropriately; error messages are clear.
        \end{itemize}
\end{enumerate}

\subsection{Non-Functional Test Cases}

\subsubsection{Performance Testing}
To check if the platform can handle high user loads, fast page loads, and real-time responses during tasks. This ensures the system remains responsive and efficient even under stress.
\begin{itemize}[leftmargin=*]
    \item Home page, login, and course page load within 2 seconds for 95\% of users
    \item Code execution returns result within 5 seconds for 99\% of cases
\end{itemize}

\subsubsection{Security Testing}
To ensure the platform protects sensitive data and enforces access controls.
\begin{itemize}[leftmargin=*]
    \item Only authenticated users can access protected endpoints
    \item Role-based access control enforced (e.g., only educators can create courses)
    \item No sensitive data stored in plain text; JWT securely signed
\end{itemize}

\subsubsection{Usability Testing}
To ensure the platform is user-friendly and supports a smooth learning experience.
\begin{itemize}[leftmargin=*]
    \item Navigation is clear and intuitive for both students and educators
    \item UI is responsive on mobile and desktop devices
\end{itemize}

\subsubsection{Reliability Testing}
To ensure the platform can recover from failures and provide meaningful error messages. This builds trust that user progress and course content remain stable and accessible.
\begin{itemize}[leftmargin=*]
    \item System recovers from AWS or DB failure without data loss
    \item Error messages are meaningful for invalid operations
\end{itemize}

\subsubsection{Compatibility Testing}
To verify that the platform works correctly across different devices, operating systems, and browsers. This ensures that all users have a consistent experience regardless of their setup.
\begin{itemize}[leftmargin=*]
    \item Supported browsers (Chrome, Firefox, Safari, Edge) render all pages correctly
    \item All main functions work on Windows, macOS, Android, iOS
\end{itemize}

\section{Resource Allocation}

\subsection{Team Roles and Responsibilities}
\begin{tabularx}{\textwidth}{l l X}
\toprule
Role & Name & Responsibilities \\
\midrule
Backend Developer & C. H. Yu & Assist with unit/integration test, bug fixing \\
Cloud Engineer & Ankhbayar & Deploy project to AWS, configure hosting and cloud resources \\
UI/UX Designer & H. T. Lei & Validate user interface and accessibility \\
Frontend Developer & H. K. Yum & Develop and maintain user interface; assist with unit/integration testing and bug fixing \\
Documentation Specialist & C. W. Leung & Maintain documentation, reports \\
Product Owner & Group A2 & Validate requirements and acceptance criteria \\
\bottomrule
\end{tabularx}

\subsection{Tools and Software}
\begin{itemize}[leftmargin=*]
    \item Test Management: GitHub issues, test cases in code repository
    \item Automation: Cypress (UI), Jest (unit), custom scripts
    \item Bug Tracking: GitHub, Jira
    \item Performance: Browser dev tools, Lighthouse
    \item Compatibility: BrowserStack, manual device testing
    \item CI/CD: GitHub Actions
\end{itemize}

\subsection{Testing Environments}
\begin{tabularx}{\textwidth}{l X l}
\toprule
Environment & Purpose & Owner \\
\midrule
Development & Unit testing, feature development & Developers \\
Staging & System/integration testing, regression & QA Team \\
Production (UAT) & Final acceptance, release candidate & Product Owner \\
\bottomrule
\end{tabularx}

\subsection{Time and Budget Estimation}
\begin{itemize}[leftmargin=*]
    \item Test Planning: 10\%
    \item Test Case Development: 15\%
    \item Test Execution: 50\%
    \item Bug Fixing/Retesting: 20\%
    \item Regression: 5\%
    \item Budget: Team time, device access, third-party tools (if any)
\end{itemize}

\section{Testing Approach}

\subsection{Types of Testing}
\begin{itemize}[leftmargin=*]
    \item \textbf{Unit Testing}: Rust/TypeScript unit tests for backend/frontend logic
    \item \textbf{Integration Testing}: End-to-end user flows (registration, course enrollment, code submission)
    \item \textbf{System Testing}: Full workflow from user registration to course completion
    \item \textbf{Regression Testing}: After each major code change
    \item \textbf{User Acceptance Testing}: Final validation by product owner/instructors
\end{itemize}

\subsection{Methodologies}
\begin{itemize}[leftmargin=*]
    \item Manual testing for UI/UX and major user stories
    \item Automated testing for regression and repetitive flows
    \item Exploratory testing for edge and negative cases
\end{itemize}

\section{Timeline and Schedule}

\subsection{Waterfall Model Example}
\begin{tabularx}{\textwidth}{l l X}
\toprule
Phase & Duration & Activities \\
\midrule
Test Planning \& Preparation & Week 1 & Test plan, environment setup \\
Unit Testing & Week 2 & Backend/frontend unit tests \\
Integration Testing & Week 3 & End-to-end flow testing \\
System Testing & Week 4 & Full platform testing \\
UAT \& Regression & Week 5 & User acceptance, bug fixing, regression \\
\bottomrule
\end{tabularx}

\subsection{Agile/Sprint Model (If Used)}
\begin{itemize}[leftmargin=*]
    \item Each sprint: Write user-story based tests, run regression, demo to stakeholders
\end{itemize}

\section{Risk Assessment and Mitigation}

\subsection{Risk Analysis}
\begin{itemize}[leftmargin=*]
    \item \textbf{Delays in Development}: Frequent communication, adjust test schedule
    \item \textbf{High Bug Volume}: Prioritize critical bugs, allocate more QA time
    \item \textbf{Integration Failures}: Early integration testing, CI/CD monitoring
    \item \textbf{Compatibility Issues}: Early device/browser testing, use BrowserStack
    \item \textbf{Untestable Features}: Add debug/logging where possible, clarify requirements
\end{itemize}

\section{Success Criteria}

\subsection{Acceptance Criteria}
\begin{itemize}[leftmargin=*]
    \item All high/critical bugs are resolved
    \item All functional and non-functional requirements pass
    \item Positive feedback from UAT/instructors
    \item System is ready for deployment
\end{itemize}

\section{Reporting Requirements}

\subsection{Reporting Methods}
\begin{itemize}[leftmargin=*]
    \item Test results and bug status tracked via GitHub issues
    \item Test execution summary and coverage report before release
    \item Weekly status meetings with stakeholders
    \item Final test summary and sign-off document
\end{itemize}

\end{document}